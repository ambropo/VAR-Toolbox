
\documentclass[11pt,a4paper]{report}
%%%%%%%%%%%%%%%%%%%%%%%%%%%%%%%%%%%%%%%%%%%%%%%%%%%%%%%%%%%%%%%%%%%%%%%%%%%%%%%%%%%%%%%%%%%%%%%%%%%%%%%%%%%%%%%%%%%%%%%%%%%%%%%%%%%%%%%%%%%%%%%%%%%%%%%%%%%%%%%%%%%%%%%%%%%%%%%%%%%%%%%%%%%%%%%%%%%%%%%%%%%%%%%%%%%%%%%%%%%%%%%%%%%%%%%%%%%%%%%%%%%%%%%%%%%%
\usepackage{makeidx}
\usepackage{amssymb}
\usepackage{amsfonts}
\usepackage{amsmath}
\usepackage[english]{babel}
\usepackage[latin1]{inputenc}
\usepackage{graphicx}
\usepackage{float}
\usepackage[singlelinecheck=false,labelsep=space,indention=.5cm,labelfont=bf,center]{caption}
\usepackage[a4paper,hmargin={3.5cm,3.5cm},vmargin={3.5cm,3cm},includefoot,bindingoffset=0cm]{geometry}
\usepackage[comma]{natbib}
\usepackage[dvipsnames,usenames]{color}
\usepackage{appendix}
\usepackage{setspace}
\usepackage{booktabs,tabularx}
\usepackage{supertabular}
\usepackage{mathpazo}
\usepackage{verbatim}
\usepackage{fancyhdr}
\usepackage{lastpage}
\usepackage{boxedminipage}
\usepackage[colorlinks,linkcolor=blue]{hyperref}

\setcounter{MaxMatrixCols}{10}
%TCIDATA{OutputFilter=LATEX.DLL}
%TCIDATA{Version=5.50.0.2953}
%TCIDATA{<META NAME="SaveForMode" CONTENT="1">}
%TCIDATA{BibliographyScheme=BibTeX}
%TCIDATA{Created=Wednesday, August 10, 2011 17:16:47}
%TCIDATA{LastRevised=Wednesday, March 11, 2020 22:50:13}
%TCIDATA{<META NAME="GraphicsSave" CONTENT="32">}
%TCIDATA{<META NAME="DocumentShell" CONTENT="Standard LaTeX\Standard LaTeX Report">}
%TCIDATA{Language=American English}
%TCIDATA{CSTFile=40 LaTeX Report.cst}

\definecolor{note}{RGB}{40,0,0}
\newtheorem{theorem}{Theorem}
\newtheorem{acknowledgement}[theorem]{Acknowledgement}
\newtheorem{algorithm}[theorem]{Algorithm}
\newtheorem{axiom}[theorem]{Axiom}
\newtheorem{case}[theorem]{Case}
\newtheorem{claim}[theorem]{Claim}
\newtheorem{conclusion}[theorem]{Conclusion}
\newtheorem{condition}[theorem]{Condition}
\newtheorem{conjecture}[theorem]{Conjecture}
\newtheorem{corollary}[theorem]{Corollary}
\newtheorem{criterion}[theorem]{Criterion}
\newtheorem{definition}[theorem]{Definition}
\newtheorem{example}[theorem]{Example}
\newtheorem{exercise}[theorem]{Exercise}
\newtheorem{lemma}[theorem]{Lemma}
\newtheorem{notation}[theorem]{Notation}
\newtheorem{problem}[theorem]{Problem}
\newtheorem{proposition}[theorem]{Proposition}
\newtheorem{remark}[theorem]{Remark}
\newtheorem{solution}[theorem]{Solution}
\newtheorem{summary}[theorem]{Summary}
\newenvironment{proof}[1][Proof]{\noindent \textbf{#1.} }{\  \rule{0.5em}{0.5em}}
\input{tcilatex}
\pagestyle{fancy}
\fancyhf{}
\lhead{\nouppercase{\rightmark}}
\rfoot{\thepage \ of \pageref{LastPage}}
\renewcommand*{\thepart}{\arabic{part}}
\setcounter{tocdepth}{2}
\setcounter{secnumdepth}{2}
\numberwithin{equation}{chapter}
\numberwithin{section}{chapter}
\setlength{\parskip}{0.2cm}
\hypersetup{
pdftitle={pdf title here},
pdfsubject={},
pdfauthor={Ambrogio Cesa Bianchi},
pdfcreator={},
pdfkeywords={},
pdfstartview=FitV,
plainpages=false,
}
\graphicspath{{./graphics/}}

\begin{document}

\title{%
%TCIMACRO{\TeXButton{----}{\rule{\textwidth}{.04cm}}}%
%BeginExpansion
\rule{\textwidth}{.04cm}%
%EndExpansion
\\
A Primer On Vector Autoregressions}
\author{{\small Ambrogio Cesa Bianchi\thanks{
\emph{Disclaimer}. These explanatory notes result from the elaboration of
text and figures found in published papers, unpublished manuscripts,
personal notes, and a lot of material found on the web that I collected over
a long time.\ If you find any lack of attribution or error in quoting the
source of the material in these notes ---as well as if you find any errors
or you would like to give me comments--- please email me at
ambrogio.cesabianchi@gmail.com.}} \and 
%TCIMACRO{\TeXButton{----}{\rule{\textwidth}{.04cm}}}%
%BeginExpansion
\rule{\textwidth}{.04cm}%
%EndExpansion
}
\maketitle
\tableofcontents

\onehalfspacing
\newpage

%TCIMACRO{%
%\TeXButton{---Note}{\begin{quotation}
%\noindent \rule{1cm}{.4pt}   {\scshape Note}   \hrulefill
%\vspace{-.36cm}
%
%\noindent \hrulefill
%\small
%\singlespacing
%\color{note}\sffamily}}%
%BeginExpansion
\begin{quotation}
\noindent \rule{1cm}{.4pt}   {\scshape Note}   \hrulefill
\vspace{-.36cm}

\noindent \hrulefill
\small
\singlespacing
\color{note}\sffamily%
%EndExpansion

What is a variance-covariance matrix? The formula for the variance of a
univariate time series $x=[x_{1},x_{2},...,x_{T}]$ is:%
\begin{equation*}
\mathsf{VAR}=\sum_{t=0}^{T}\frac{\left( x_{t}-\bar{x}\right) ^{2}}{N}%
=\sum_{t=0}^{T}\frac{\left( x_{t}-\bar{x}\right) \left( x_{t}-\bar{x}\right) 
}{N}.
\end{equation*}%
If we have a bivariate time series, such as:%
\begin{equation*}
\mathbf{x}_{t}=\left[ 
\begin{array}{cccc}
x_{1} & x_{2} & ... & x_{T} \\ 
x_{1} & x_{2} & ... & x_{T}%
\end{array}%
\right] ,
\end{equation*}%
the formula simply becomes:%
\begin{equation*}
\mathsf{VCV}=\left[ 
\begin{array}{cc}
\sum_{t=0}^{T}\frac{\left( x_{t}-\bar{x}\right) \left( x_{t}-\bar{x}\right) 
}{N} & \sum_{t=0}^{T}\frac{\left( x_{t}-\bar{x}\right) \left( x_{t}-x\right) 
}{N} \\ 
\sum_{t=0}^{T}\frac{\left( x_{t}-x\right) \left( x_{t}-\bar{x}\right) }{N} & 
\sum_{t=0}^{T}\frac{\left( x_{t}-x\right) \left( x_{t}-x\right) }{N}%
\end{array}%
\right] =\left[ 
\begin{array}{cc}
\mathsf{VAR}(x) & \mathsf{COV}(x,x) \\ 
\mathsf{COV}(x,x) & \mathsf{VAR}(x)%
\end{array}%
\right] .
\end{equation*}

%TCIMACRO{%
%\TeXButton{---Note End}{\color{black}
%\noindent \hrulefill 
%\vspace{-.5cm}
%
%\noindent \hrulefill
%\end{quotation}
%\bigskip}}%
%BeginExpansion
\color{black}
\noindent \hrulefill 
\vspace{-.5cm}

\noindent \hrulefill
\end{quotation}
\bigskip%
%EndExpansion

\noindent Note that the basic VAR(1) model may be too poor to sufficiently
summarize the main characteristics of the data. Deterministic terms (such as
time trend or seasonal dummy variables) and exogenous variables (such as the
price of oil) can be added to the basic specification. The general form of
the \textbf{stationary structural VAR(p)} model with deterministic terms ($%
\mathbf{Z}_{t}$) and exogenous variables ($\mathbf{W}_{t}$) is given by:%
\begin{equation*}
\mathbf{Ax}_{t}=\mathbf{B}_{1}\mathbf{x}_{t-1}+\mathbf{B}_{2}\mathbf{x}%
_{t-2}+...+\mathbf{B}_{p}\mathbf{x}_{t-p}+\boldsymbol{\Lambda }\mathbf{Z}%
_{t}+\boldsymbol{\Psi }\mathbf{W}_{t}+\mathbf{\varepsilon }_{t}.
\end{equation*}

\section{Estimation}

The previous section showed that structural VARs are our ultimate object of
interest. However, the estimation of a structural VAR with OLS violates one
of the assumption of the classical regression models, i.e. the regressors
and the error terms have to be orthogonal. $%
a_{11}x_{1t}+a_{12}x_{2t}=b_{11}x_{1,t-1}+b_{12}x_{2,t-1}+\varepsilon _{1t},$

To see that, let's consider the second equation of the system in (\ref%
{eq:var_struct_a}):%
\begin{equation*}
a_{22}x_{2t}=-a_{21}x_{1t}+b_{21}x_{1,t-1}+b_{22}x_{2,t-1}+\varepsilon _{2t}.
\end{equation*}%
Below, I will compute the covariance between $x_{1t}$ (one of the
regressors) and $\varepsilon _{2t}$ (the error term), and show that is not
zero. Our object of interest is $\mathsf{COV}\left[ x_{1t},\varepsilon _{2t}%
\right] $: note that we can substitute for $x_{1t}$ using the first equation
of the system in (\ref{eq:var_struct_a}) and get:%
%TCIMACRO{\TeXButton{ReduceSize_Start}{\footnotesize}}%
%BeginExpansion
\footnotesize%
%EndExpansion
\begin{equation*}
\mathsf{COV}\left[ \underset{\text{from the first equation of (\ref%
{eq:var_struct_a})}}{\underbrace{-\frac{a_{12}}{a_{11}}x_{2t}+\frac{b_{11}}{%
a_{11}}x_{1,t-1}+\frac{b_{12}}{a_{11}}x_{2,t-1}+\frac{\varepsilon _{1t}}{%
a_{11}}}},\varepsilon _{2t}\right] .
\end{equation*}%
%TCIMACRO{\TeXButton{ReduceSize_End}{\normalsize}}%
%BeginExpansion
\normalsize%
%EndExpansion
But from the second equation of the system in (\ref{eq:var_struct_a}) it
follows that:%
%TCIMACRO{\TeXButton{ReduceSize_Start}{\footnotesize} }%
%BeginExpansion
\footnotesize
%EndExpansion
\begin{equation*}
\mathsf{COV}\left[ -\frac{a_{12}}{a_{11}}\left( \underset{\text{from the
second equation of (\ref{eq:var_struct_a})}}{\underbrace{-\frac{a_{21}}{%
a_{22}}x_{1t}+\frac{b_{21}}{a_{22}}x_{1,t-1}+\frac{b_{22}}{a_{22}}x_{2,t-1}+%
\frac{\varepsilon _{2t}}{a_{22}}}}\right) +\frac{b_{11}}{a_{11}}x_{1,t-1}+%
\frac{b_{12}}{a_{11}}x_{2,t-1}+\frac{\varepsilon _{1t}}{a_{11}},\varepsilon
_{2t}\right] ,
\end{equation*}%
%TCIMACRO{\TeXButton{ReduceSize_End}{\normalsize}}%
%BeginExpansion
\normalsize%
%EndExpansion
which implies that: 
\begin{equation*}
\mathsf{COV}\left[ x_{1t},\varepsilon _{2t}\right] =\frac{a_{12}}{a_{11}}%
\frac{a_{21}}{a_{22}}\mathsf{COV}\left[ x_{1t},\varepsilon _{2t}\right] -%
\frac{a_{12}}{a_{11}a_{22}}\mathsf{COV}\left[ \varepsilon _{2t},\varepsilon
_{2t}\right] .
\end{equation*}%
From the last equality it follows that 
\begin{equation*}
\mathsf{COV}[x_{1t},\varepsilon _{2t}]=\frac{a_{12}}{%
a_{12}a_{21}-a_{11}a_{22}}\neq 0
\end{equation*}%
This is a violation of the assumptions of the classical regression model.
Therefore, OLS estimates of equation for $x_{2t}$ will yield inconsistent
estimates of the parameters of the model unless we assume that $a_{12}=0$.
The same happens if we want to estimate the structural equation for $x_{1t}$%
, unless we assume $a_{21}=0.$

The reduced form VAR\ does not suffer from the same problem and can be
estimated by OLS. However, as we have seen above, the reduced-form errors
terms will correlated and, therefore, it will not be possible to evaluate
the effect of the structural shock on the system.

The OLS\ estimates of the parameters of a\ reduced form VAR(1)\ are:%
\begin{eqnarray*}
\mathbf{F} &=&(\mathbf{x}_{t-1}\mathbf{x}_{t-1}^{\prime })^{-1}\mathbf{x}%
_{t-1}\mathbf{x}_{t}^{\prime }, \\
\mathbf{\hat{u}}_{t} &=&\mathbf{x}_{t}-\mathbf{\hat{F}y}_{t-1}, \\
\mathbf{\hat{\Sigma}}_{u} &=&\mathsf{E}\left[ \mathbf{\hat{u}}_{t}\mathbf{%
\hat{u}}_{t}^{\prime }\right] =\frac{\mathbf{\hat{u}}_{t}\mathbf{\hat{u}}%
_{t}^{\prime }}{T-1}.
\end{eqnarray*}%
Note that, in general, $\mathbf{\hat{\Sigma}}_{u}$, is a symmetric
non-diagonal matrix:%
\begin{equation*}
\mathbf{\hat{\Sigma}}_{u}=\left[ 
\begin{array}{cc}
\sigma _{1}^{2} & \sigma _{12}^{2} \\ 
- & \sigma _{2}^{2}%
\end{array}%
\right] .
\end{equation*}%
Its diagonal elements are the variances of the estimated reduced-form error
terms, $\sigma _{1}^{2}$ and $\sigma _{2}^{2}$. The off-diagonal elements
are instead and the covariance between the estimated error terms ($\sigma
_{12}=\sigma _{21}$). The covariance between the estimated error terms plays
an important role in estimated reduced form VARs because it collects the
information on the contemporaneous interaction of the variables in the
system.

\subsection{Estimation with the VAR\ Toolbox}

Before seeing how we can estimate a VAR\ in MatLab, note that In MatLab we
generally use a different notation. Specifically, we generally write the two
time series $x_{1t}$ and $x_{2t}$ in a matrix \textbf{$x$} of dimension $%
T\times 2$ (i.e., including $2$ time series with $T$ observations each). In
other words, we can write:%
\begin{equation*}
\mathbf{x}=\left[ 
\begin{array}{cc}
x_{11} & x_{21} \\ 
x_{12} & x_{22} \\ 
... & ... \\ 
x_{1T} & x_{2T}%
\end{array}%
\right] .
\end{equation*}%
Therefore we can write the system of equations in (\ref{eq:var_red_a}) as:%
\begin{equation}
\left[ 
\begin{array}{cc}
x_{1t} & x_{2t}%
\end{array}%
\right] =\left[ 
\begin{array}{cc}
x_{1t-1} & x_{2t-1}%
\end{array}%
\right] \left[ 
\begin{array}{cc}
f_{11} & f_{21} \\ 
f_{12} & f_{22}%
\end{array}%
\right] +\left[ 
\begin{array}{cc}
u_{1t} & u_{2t}%
\end{array}%
\right] ,  \label{eq:var_red_b_matlab}
\end{equation}%
which clearly yields%
\begin{equation*}
\left\{ 
\begin{array}{c}
x_{1t}=f_{11}x_{1,t-1}+f_{12}x_{2,t-1}+u_{1t}, \\ 
x_{2t}=f_{21}x_{1,t-1}+f_{22}x_{2,t-1}+u_{2t},%
\end{array}%
\right.
\end{equation*}%
as in equation (\ref{eq:var_red_a}). Even though the matricial notation
above is a different from the on used above, the reduced-form representation
is of course the same. According to our above notation, equation \ref%
{eq:var_red_b_matlab} can be also expressed as:%
\begin{equation}
\mathbf{x}_{t}=\mathbf{x}_{t-1}\mathbf{F}^{\prime }+\mathbf{u}_{t}.
\label{eq:var_red_c_matlab}
\end{equation}

%TCIMACRO{%
%\TeXButton{---Note}{\begin{quotation}
%\noindent \rule{1cm}{.4pt}   {\scshape Note}   \hrulefill
%\vspace{-.36cm}
%
%\noindent \hrulefill
%\small
%\singlespacing
%\color{note}\sffamily}}%
%BeginExpansion
\begin{quotation}
\noindent \rule{1cm}{.4pt}   {\scshape Note}   \hrulefill
\vspace{-.36cm}

\noindent \hrulefill
\small
\singlespacing
\color{note}\sffamily%
%EndExpansion

In Matlab, to perform the estimation of a reduced form VAR you can use the
procedure \texttt{VARmodel.m}, whose description is provided below.\medskip

\texttt{function VARout = VARmodel(DATA,nlag,c\_case,DATA\_EX,nlag\_ex)}

\texttt{\% ===============================================================}

\texttt{\% Performs vector autogressive estimation}

\texttt{\% ===============================================================}

\texttt{\% VARout = VARmodel(DATA,nlag,c\_case,DATA\_EX)}

\texttt{\% ---------------------------------------------------------------}

\texttt{\% INPUTS }

\texttt{\% DATA = an (nobs x neqs) matrix of y-vectors}

\texttt{\% nlag = the lag length}

\texttt{\%}

\texttt{\% OPTIONAL INPUTS}

\texttt{\% c\_case = 0 no constant; 1 with constant ; 2 constant + }

\%\qquad \qquad \qquad\ \ \texttt{time trend}

\texttt{\% DATA\_EX = optional matrix of variables (nobs x nvar\_ex)}

\texttt{\% nlag\_ex = number of lags for exogeonus variables }

\%\qquad \qquad \qquad\ \ \ \texttt{(default = 0)}

\texttt{\% ===============================================================}%
\medskip

%TCIMACRO{%
%\TeXButton{---Note End}{\color{black}
%\noindent \hrulefill 
%\vspace{-.5cm}
%
%\noindent \hrulefill
%\end{quotation}
%\bigskip}}%
%BeginExpansion
\color{black}
\noindent \hrulefill 
\vspace{-.5cm}

\noindent \hrulefill
\end{quotation}
\bigskip%
%EndExpansion

The \texttt{DATA} matrix should be such that each column is a variable and
each row is an observation (with no missing observations), as the following
one:%
\begin{equation*}
\mathtt{DATA}=\mathbf{x}^{\prime }=\left[ 
\begin{array}{cc}
x_{11} & x_{21} \\ 
x_{12} & x_{22} \\ 
... & ... \\ 
x_{1T} & x_{2T}%
\end{array}%
\right] .
\end{equation*}%
The other two inputs that are needed for the estimation are the number of
lags and the deterministic variables (constant and trend). For example, to
estimate a bivariate VAR$(2)$ with constant and trend on the the matrix 
\textbf{$x$} you can type in Matlab:\medskip 

\texttt{VARout = VARmodel(Y,2,2);}\medskip

\noindent The \texttt{VARmodel}\ code estimates the reduced form VAR\ and
saves the results in the \texttt{VARout} structure. For example, a bivariate
VAR$(2)$\ with trend and constant can be written as:%
\begin{equation*}
\begin{array}{c}
x_{1t}=\alpha _{1}+\beta
_{1}tr+f_{11}x_{1t-1}+f_{12}x_{2t-1}+f_{11}^{2}x_{1t-2}+f_{12}^{2}x_{2t-2}+u_{1t},
\\ 
x_{1t}=\alpha _{2}+\beta
_{2}tr+f_{21}x_{1t-1}+f_{22}x_{2t-1}+f_{21}^{2}x_{1t-2}+f_{22}^{2}x_{2t-2}+u_{2t},%
\end{array}%
\end{equation*}%
or as in equation (\ref{eq:var_red_b_matlab}):%
\begin{equation*}
\left[ 
\begin{array}{cc}
x_{1t} & x_{2t}%
\end{array}%
\right] =\left[ 
\begin{array}{cccccc}
1 & tr & x_{1t-1} & x_{2t-1} & x_{1t-2} & x_{2t-2}%
\end{array}%
\right] \underset{\mathbf{F}^{\prime }}{\underbrace{\left[ 
\begin{array}{cc}
\alpha _{1} & \alpha _{2} \\ 
\beta _{1} & \beta _{2} \\ 
f_{11} & f_{21} \\ 
f_{12} & f_{22} \\ 
f_{11}^{2} & f_{21}^{2} \\ 
f_{12}^{2} & f_{22}^{2}%
\end{array}%
\right] .}}
\end{equation*}%
The matrix \texttt{VARout.beta} collects the estimated coefficients as in
the equation above. Therefore, we have that:%
\begin{equation*}
\mathtt{VARout.beta}=\mathbf{F}^{\prime }.
\end{equation*}%
The \texttt{VARout} structure includes many additional standard results of
VAR\ analysis. I refer to the \texttt{VARmodel.m}\ file for a full
description of those results.

\subsection{A simple example}

We can now estimate the VAR with some real data. We use UK\ quarterly data
on GDP growth, inflation and, short term interest rates from 1985.I to
2013.III, plotted in the Figure below.

%TCIMACRO{%
%\TeXButton{FIG}{\begin{figure}[!ht]\vspace{.1cm}
%\centering
%\begin{minipage}[b]{0.99\textwidth}}}%
%BeginExpansion
\begin{figure}[!ht]\vspace{.1cm}
\centering
\begin{minipage}[b]{0.99\textwidth}%
%EndExpansion
%TCIMACRO{%
%\TeXButton{figure}{\includegraphics[width=.30\textwidth]{X1.pdf}
%\includegraphics[width=.30\textwidth]{X2.pdf}
%\includegraphics[width=.30\textwidth]{X3.pdf}
%\vspace{.3cm}}}%
%BeginExpansion
\includegraphics[width=.30\textwidth]{X1.pdf}
\includegraphics[width=.30\textwidth]{X2.pdf}
\includegraphics[width=.30\textwidth]{X3.pdf}
\vspace{.3cm}%
%EndExpansion
\caption{ {\scshape Data: GDP Growth, Inflation, Short-term Interest Rate}}%
%TCIMACRO{%
%\TeXButton{FIG_END}{\end{minipage}
%\end{figure}}}%
%BeginExpansion
\end{minipage}
\end{figure}%
%EndExpansion

\noindent We estimate, as an example, a VAR(1) with a constant (namely, $%
\mathbf{x}_{t}=\mathbf{c}+\mathbf{Fx}_{t-1}+\mathbf{u}_{t}$). The typical
VAR output is given by a matrix of coefficients ($\mathbf{F}^{\prime }$):

\begin{center}
%TCIMACRO{%
%\TeXButton{table}{\begin{tabular}{lrrr}
%\addlinespace \toprule & Real GDP & GDP Deflator & Policy Rate \\ 
%\midrule c & 1.14 & 1.19 & -0.11 \\ 
%Real GDP(-1) & 0.61 & -0.07 & 0.06 \\ 
%GDP Deflator(-1) & -0.09 & 0.02 & 0.03 \\ 
%Policy Rate(-1) & 0.01 & 0.30 & 0.96 \\ 
%\bottomrule &  &  & 
%\end{tabular}}}%
%BeginExpansion
\begin{tabular}{lrrr}
\addlinespace \toprule & Real GDP & GDP Deflator & Policy Rate \\ 
\midrule c & 1.14 & 1.19 & -0.11 \\ 
Real GDP(-1) & 0.61 & -0.07 & 0.06 \\ 
GDP Deflator(-1) & -0.09 & 0.02 & 0.03 \\ 
Policy Rate(-1) & 0.01 & 0.30 & 0.96 \\ 
\bottomrule &  &  & 
\end{tabular}%
%EndExpansion
\end{center}

\noindent and the correlation matrix of the reduced-form residuals:

\begin{center}
%TCIMACRO{%
%\TeXButton{table}{\begin{tabular}{lrrr}
%\addlinespace \toprule & Real GDP & GDP Deflator & Policy Rate \\ 
%\midrule Real GDP & 1.000 & -0.178 & 0.373 \\ 
%GDP Deflator & -- & 1.000 & 0.137 \\ 
%Policy Rate & -- & -- & 1.000 \\ 
%\bottomrule &  &  & 
%\end{tabular}}}%
%BeginExpansion
\begin{tabular}{lrrr}
\addlinespace \toprule & Real GDP & GDP Deflator & Policy Rate \\ 
\midrule Real GDP & 1.000 & -0.178 & 0.373 \\ 
GDP Deflator & -- & 1.000 & 0.137 \\ 
Policy Rate & -- & -- & 1.000 \\ 
\bottomrule &  &  & 
\end{tabular}%
%EndExpansion
\end{center}

After estimating the VAR\ we should check the goodness of our model. We do
not cover this in detail but before interpreting the VAR results we should
check a number of assumptions. Loosely speaking we need to check that the
reduced-form residuals are, normally distributed, not autocorrelated, and
not heteroskedastic (i.e., have constant variance). Finally, we need to
check that the VAR is stationary (we'll see in a second what it means). But
before moving to stationarity, why do we need to check the residuals? The
reason is quite simple. The VAR believes that:%
\begin{equation*}
\mathbf{x}\sim \mathcal{N}(\mathbf{\mu },\mathbf{\sigma })\ \ \
\Longrightarrow \ \ \ \left\{ 
\begin{array}{lll}
\Delta y & \sim & \mathcal{N}(\mu ^{\Delta y},\sigma ^{\Delta y}) \\ 
\pi & \sim & \mathcal{N}(\mu ^{\pi },\sigma ^{\pi }) \\ 
i & \sim & \mathcal{N}(\mu ^{i},\sigma ^{i})%
\end{array}%
\right.
\end{equation*}%
If the data that we feed into the VAR\ has not these features, the residuals
will inherit them. Moreover note that:

\begin{itemize}
\item Mean ($\mu $) and variance\ ($\sigma $) are constant $\Longrightarrow $
the data have to be stationary!

\item The mean ($\mu $) and variance\ ($\sigma $) are not known a priori
\end{itemize}

\noindent Can we recover the mean of our endogenous variables? Yes. It is
given by $\mathbf{\mu }=\mathbb{E}[\mathbf{x}_{t}]$%
\begin{equation*}
\begin{array}{lll}
\mathsf{E}[\mathbf{x}_{t}] & = & \mathsf{E}[\mathbf{c}+\mathbf{Fx}_{t-1}+%
\mathbf{u}_{t}]=\smallskip \\ 
& = & \mathsf{E}[\mathbf{c+F(c+Fx}_{t-2}\mathbf{+u}_{t-1}\mathbf{)+u}_{t}%
\mathbf{]=E[c+Fc+F}^{2}\mathbf{x}_{t-2}\mathbf{+Fu}_{t-1}\mathbf{+u}_{t}%
\mathbf{]=}\smallskip \\ 
& = & \mathsf{E}[\mathbf{c+Fc+F}^{2}\mathbf{c+...+F}^{t-2}\mathbf{c+F}^{t-1}%
\mathbf{x}_{1}\mathbf{+F}^{t-2}\mathbf{u}_{2}\mathbf{+...+F}^{2}\mathbf{u}%
_{t-2}\mathbf{+Fu}_{t-1}\mathbf{+u}_{t}]=\smallskip \\ 
& = & ...\ =\smallskip \\ 
& = & \mathsf{E}\left[ \sum\limits_{j=0}^{t-2}\mathbf{F}^{j}\mathbf{c}+%
\mathbf{F}^{t-1}\mathbf{x}_{1}+\sum\limits_{j=0}^{t-2}\mathbf{F}^{j}\mathbf{u%
}_{t-j}\right]%
\end{array}%
\end{equation*}%
If VAR\ is stationary (wait one more minute to see what it means), from the
last expression we get:%
\begin{equation*}
\mathsf{E}[\mathbf{x}_{t}]=\left( \mathbf{I}-\mathbf{F}\right) ^{-1}\mathbf{c%
}=\mathbf{\mu }
\end{equation*}

%TCIMACRO{%
%\TeXButton{---Note}{\begin{quotation}
%\noindent \rule{1cm}{.4pt}   {\scshape Note}   \hrulefill
%\vspace{-.36cm}
%
%\noindent \hrulefill
%\small
%\singlespacing
%\color{note}\sffamily}}%
%BeginExpansion
\begin{quotation}
\noindent \rule{1cm}{.4pt}   {\scshape Note}   \hrulefill
\vspace{-.36cm}

\noindent \hrulefill
\small
\singlespacing
\color{note}\sffamily%
%EndExpansion

Geometric series. Consider the following sum $\sum\limits_{j=0}^{T}Y^{j}$.
When $T\rightarrow \infty $ we have: 
\begin{equation*}
\mathbf{(1+Y}+\mathbf{Y}^{2}+\mathbf{...}+\mathbf{Y}^{\infty }\mathbf{)=}%
\left( \mathbf{I}-\mathbf{Y}\right) ^{-1}
\end{equation*}%
if and only if:

\begin{itemize}
\item $\left\vert eig(\mathbf{Y)}\right\vert <1$ when $Y$ is a matrix

\item $Y<1$ when $Y$ is a number
\end{itemize}

\noindent Therefore, for $T$ large enough we have 
\begin{eqnarray*}
\mathsf{E}[\mathbf{x}_{t}] &=&\mathsf{E}\left[ \sum\limits_{j=0}^{t-2}%
\mathbf{F}^{j}\mathbf{c}+\mathbf{F}^{t-1}\mathbf{x}_{0}+\sum%
\limits_{j=0}^{t-2}\mathbf{F}^{j}\mathbf{u}_{t-j}\right] = \\
&=&\mathsf{E}\left[ \left( \mathbf{I}-\mathbf{F}\right) ^{-1}\mathbf{c}%
\right] +\underset{\approx 0}{\underbrace{\mathbf{F}^{t-1}\mathbf{x}_{1}}}+%
\underset{\approx 0}{\underbrace{\mathsf{E}\left[ \sum\limits_{j=0}^{t-2}%
\mathbf{F}^{j}\mathbf{u}_{t-j}\right] }}
\end{eqnarray*}

%TCIMACRO{%
%\TeXButton{---Note End}{\color{black}
%\noindent \hrulefill 
%\vspace{-.5cm}
%
%\noindent \hrulefill
%\end{quotation}
%\bigskip}}%
%BeginExpansion
\color{black}
\noindent \hrulefill 
\vspace{-.5cm}

\noindent \hrulefill
\end{quotation}
\bigskip%
%EndExpansion

\noindent A VAR\ is stationary (or stable) when: 
\begin{equation*}
\left\vert eig(\mathbf{F)}\right\vert <1.
\end{equation*}%
In this case, the VAR thinks that $(\mathbf{I}-\mathbf{F})^{-1}\mathbf{c}$
is the unconditional mean of the stochastic processes governing our
variables. Note that the unconditional mean is an interesting element of
VAR\ analysis but it is often ignored. If we recover both $\mathbf{c}$ and $%
\mathbf{F}$ from the estimated VAR, we can compute the unconditional mean as 
\begin{equation*}
\begin{array}{c}
\mathbf{(I-\mathbf{F})}^{-1}\mathbf{c}=\left[ 
\begin{array}{ccc}
2.32 & -0.28 & -1.89 \\ 
1.34 & 1.24 & 10.58 \\ 
4.89 & 0.67 & 33.87%
\end{array}%
\right] \left[ 
\begin{array}{c}
1.14 \\ 
1.19 \\ 
-0.11%
\end{array}%
\right] =\left[ 
\begin{array}{c}
2.51 \\ 
1.80 \\ 
2.49%
\end{array}%
\right] =\left[ 
\begin{array}{l}
\mu ^{\Delta y} \\ 
\mu ^{\pi } \\ 
\mu ^{i}%
\end{array}%
\right]%
\end{array}%
\end{equation*}%
The stationarity of the data is important because in absence of shocks, each
variable will converge to its unconditional mean. For example, if we start
from an hypothetical point in time $T$ where $x_{T}=2\%$, $\pi _{T}=2\%$,
and $r_{T}=4\%$, we would have:

%TCIMACRO{%
%\TeXButton{FIG}{\begin{figure}[!ht]\vspace{.1cm}
%\centering
%\begin{minipage}[b]{0.79\textwidth}}}%
%BeginExpansion
\begin{figure}[!ht]\vspace{.1cm}
\centering
\begin{minipage}[b]{0.79\textwidth}%
%EndExpansion
%TCIMACRO{%
%\TeXButton{figure}{\includegraphics[width=.99\textwidth]{uncond_mean.pdf}
%\vspace{.3cm}}}%
%BeginExpansion
\includegraphics[width=.99\textwidth]{uncond_mean.pdf}
\vspace{.3cm}%
%EndExpansion
\caption{ {\scshape Convergence To The Unconditional Mean}}\label{fig:}%
%TCIMACRO{%
%\TeXButton{FIG_END}{\end{minipage}
%\end{figure}}}%
%BeginExpansion
\end{minipage}
\end{figure}%
%EndExpansion

\chapter{THE IDENTIFICATION PROBLEM}

So far we have estimated our reduced form VAR. But what can we do with it?

What are the dynamic properties of these variables? [{Look at lagged
coefficients}]

How do these variables interact? [{Look at cross-variable coefficients}]

What will be inflation tomorrow [{Forecasting}]

What is the effect of a monetary policy shock on GDP and inflation? [{???}]

\noindent The problem with reduced-form VARs is that they do not tell us
anything about the structure of the economy. In particular, we cannot
interpret the reduced-form error terms ($\mathbf{u}$) as {structural shocks.
Imagine that you have to }interpret a movement in $u_{r}$? Since it is a
linear combination of $\varepsilon _{r}$, $\varepsilon _{\Delta y}$, and $%
\varepsilon _{\pi }$ it is hard to know what is the nature of the shock: is
it a shock to aggregate demand that induces policy to move the interest
rate?\ Or is it a monetary policy shock? This is the very nature of the {%
identification problem}

To answer this question we need to get back to the structural representation
(where the error terms are uncorrelated). In practice, in a reduced form VAR
we estimate$\ \mathbf{F}$ and $\mathbf{\Sigma }_{u}$, but we cannot easily
revert to the $\mathbf{A}$, the $\mathbf{B}$, and the $\mathbf{\Sigma }%
_{\varepsilon }$ that are our ultimate object of interest. So how can we
recover the structural parameters of $\mathbf{A}$, $\mathbf{B}$, and $%
\mathbf{\Sigma }_{\varepsilon }$?

From equation (\ref{eq:var_reduced_c}) we know that:%
\begin{eqnarray}
\mathbf{F} &=&\mathbf{A}^{-1}\mathbf{B},  \label{eq:1} \\
\mathbf{\hat{\Sigma}}_{u} &=&\mathsf{E}\left[ \mathbf{\hat{u}}_{t}\mathbf{%
\hat{u}}_{t}^{\prime }\right] =\mathsf{E}\left[ \mathbf{A}^{-1}\mathbf{%
\varepsilon }\left( \mathbf{A}^{-1}\mathbf{\varepsilon }\right) ^{\prime }%
\right] =\mathbf{A}^{-1}\mathbf{\Sigma }_{\varepsilon }\left( \mathbf{A}%
^{-1}\right) ^{\prime }=\mathbf{A}^{-1}\mathbf{A}^{-1\prime }.  \label{eq:2}
\end{eqnarray}%
Therefore, if we recover $\mathbf{A}$ from equation (\ref{eq:2}), we will be
able to recover $\mathbf{B}$ from equation (\ref{eq:1}).

%TCIMACRO{%
%\TeXButton{---Note}{\begin{quotation}
%\noindent \rule{1cm}{.4pt}   {\scshape Note}   \hrulefill
%\vspace{-.36cm}
%
%\noindent \hrulefill
%\small
%\singlespacing
%\color{note}\sffamily}}%
%BeginExpansion
\begin{quotation}
\noindent \rule{1cm}{.4pt}   {\scshape Note}   \hrulefill
\vspace{-.36cm}

\noindent \hrulefill
\small
\singlespacing
\color{note}\sffamily%
%EndExpansion

The variance--covariance matrix of residuals. Why $\Sigma _{\mathbf{u}}=%
\mathsf{E}\left[ \mathbf{u}_{t}\mathbf{u}_{t}^{\prime }\right] ?$ The
formula for the variance of a univariate time series is $%
x=[x_{0},x_{1},...,x_{T}]:$%
\begin{equation*}
\mathsf{VAR}=\sum_{t=0}^{T}\frac{\left( x_{t}-\bar{x}\right) ^{2}}{N}.
\end{equation*}%
But the residuals (both the $\mathbf{u}_{t}$ and the $\mathbf{\varepsilon }%
_{t}$) have zero mean which implies that formula would be: 
\begin{equation*}
\mathsf{VAR}=\sum_{t=0}^{T}\frac{x_{t}^{2}}{N}.
\end{equation*}%
In a bivariate VAR\ we would have:%
\begin{equation*}
\mathbf{u}_{t}\mathbf{u}_{t}^{\prime }=\left[ 
\begin{array}{cccc}
{\footnotesize u}_{1}^{1} & {\footnotesize u}_{2}^{1} & {\footnotesize ...}
& {\footnotesize u}_{T}^{1} \\ 
{\footnotesize u}_{1}^{2} & {\footnotesize u}_{2}^{2} & {\footnotesize ...}
& {\footnotesize u}_{T}^{2}%
\end{array}%
\right] \left[ 
\begin{array}{cc}
{\footnotesize u}_{1}^{1} & {\footnotesize u}_{1}^{2} \\ 
{\footnotesize u}_{2}^{1} & {\footnotesize u}_{2}^{2} \\ 
{\footnotesize ...} & {\footnotesize ...} \\ 
{\footnotesize u}_{T}^{1} & {\footnotesize u}_{T}^{2}%
\end{array}%
\right] =\left[ 
\begin{array}{cc}
\sum_{t=0}^{T}\left( u_{t}^{1}\right) ^{2} & \sum_{t=0}^{T}\left(
u_{t}^{1}u_{t}^{2}\right) \\ 
- & \sum_{t=0}^{T}\left( u_{t}^{2}\right) ^{2}%
\end{array}%
\right] .
\end{equation*}%
And therefore 
\begin{equation*}
\mathsf{E}\left[ \mathbf{u}_{t}\mathbf{u}_{t}^{\prime }\right] =\left[ 
\begin{array}{cc}
\mathsf{VAR}\left[ u^{1}\right] & \mathsf{COV}\left[ u^{1}u^{2}\right] \\ 
- & \mathsf{VAR}\left[ u^{1}\right]%
\end{array}%
\right]
\end{equation*}

%TCIMACRO{%
%\TeXButton{---Note End}{\color{black}
%\noindent \hrulefill 
%\vspace{-.5cm}
%
%\noindent \hrulefill
%\end{quotation}
%\bigskip}}%
%BeginExpansion
\color{black}
\noindent \hrulefill 
\vspace{-.5cm}

\noindent \hrulefill
\end{quotation}
\bigskip%
%EndExpansion

\noindent We can think of (\ref{eq:2}) as a system of nonlinear equations in
the unknown parameters $a_{11},a_{22},a_{12}$, and $a_{21}$. More in general
the matrix $\mathbf{A}$ would contain $k\times k$ unknown parameters.
However, the $\mathbf{\Sigma }_{u}$ matrix ---given its symmetric nature---
would contain only $k(k\times 1)/2$ parameters. In other words, we have 
\begin{equation*}
\mathbf{\Sigma }_{u}=\left[ 
\begin{array}{cc}
\sigma _{1}^{2} & \sigma _{12}^{2} \\ 
- & \sigma _{2}^{2}%
\end{array}%
\right] =\frac{1}{a_{11}a_{22}-a_{12}a_{21}}%
\begin{bmatrix}
a_{22} & -a_{12} \\ 
-a_{21} & a_{11}%
\end{bmatrix}%
\begin{bmatrix}
a_{22} & -a_{21} \\ 
-a_{12} & a_{11}%
\end{bmatrix}%
=\mathbf{A}^{-1}\mathbf{A}^{-1\prime }
\end{equation*}%
which shows that there are 4 unknowns ($a_{11},a_{22},a_{12},a_{21}$) but
only 3 equations (corresponding to $\sigma _{1}^{2}$, $\sigma _{12}^{2}$, $%
\sigma _{2}^{2}$). The system is clearly under-identified, meaning that we
need at least one additional condition if we want to recover the structural
parameters. Identification can be achieved in different ways:

\begin{enumerate}
\item ZERO SHORT-RUN\ RESTRICTIONS (or recursive VAR). In a recursive VAR
identification is achieved by making the assumption on the contemporaneous
relations among variables. Since non-diagonal elements of the $\mathbf{A}$
matrix represent the contemporaneous relation between the \textbf{$x$}$_{t}$%
, this is equivalent to making assumption about the non-diagonal elements of
the $\mathbf{A}$ matrix. Specifically, we will assume that the first
variable is not affected contemporaneously by the others, the second is
contemporaneously affected by the first variable but not by the others, the
third is contemporaneously affected by the first and the second and not by
the others, and so on. In this way the error term in each regression is
uncorrelated with the error in the preceding equations. The implementation
of this method can be done by 1) estimating the equations of the VAR by
carefully including in some of the equations the contemporaneous values of
other variables as regressors or 2) applying a Cholesky decomposition to the
error terms of a reduced form VAR (as we shall see below)

\item STRUCTURAL VAR. It uses economic theory to sort out the
contemporaneous relationships between the variables (the elements of matrix $%
A$). For instance, we might know from economic theory that $a_{12}$ is equal
to $0.2$.

\item ZERO LONG-RUN RESTRICTIONS. Similarly to the short-run restrictions,
identification is achieved by making the assumption that some variables of
the VAR\ cannot affect some other variables in the long-run. Specifically we
will assume that the first variable is not affected in the long run by the
others; the second is affected in the long run by the first variable but not
by the others, and so on and so forth.

\item SIGN RESTRICTIONS. Identification is achieved by restricting the sign
of the responses of selected model variables to structural shocks, using
economic theory as a guidance
\end{enumerate}

\section{Zero short-run restrictions}

This is the methodology proposed by \cite{Sims1980}. In the context of our
bi-variate VAR(1), it consists in assuming that one variable has NO
contemporaneous effect on the other one. For example, assume that $x_{2t}$
is contemporaneously affected by $x_{1t}$ but not vice-versa. Thus we assume
that $a_{12}=0$.

A different way to express the same idea is to say that the structural
innovation $\varepsilon _{1t}$ affects both $x_{1t}$ and $x_{2t}$; while the
structural innovation $\varepsilon _{2t}$ affects only $x_{2t}$.\footnote{%
Will see that this corresponds to a triangular decomposition of the error
term, also called Cholesky decomposition.} With this assumption, we now have
3 equations and 3 unknown structural parameters, and SVAR is exactly
identified.

\begin{quotation}
{%
%TCIMACRO{%
%\TeXButton{Assumption}{\noindent {\color{Orange} \textbf{Assumption}}}}%
%BeginExpansion
\noindent {\color{Orange} \textbf{Assumption}}%
%EndExpansion
.} In what follows we assume that $x_{2t}$ is contemporaneously affected by $%
x_{1t}$ but not vice-versa. Thus we assume that $a_{12}=0.$
\end{quotation}

This is equivalent to assume that A is lower triangular:%
\begin{equation*}
\left[ 
\begin{array}{cc}
a_{11} & {%
%TCIMACRO{\TeXButton{note}{\color{red}}}%
%BeginExpansion
\color{red}%
%EndExpansion
{0}} \\ 
a_{21} & a_{22}%
\end{array}%
\right] 
\begin{bmatrix}
x_{1t} \\ 
x_{2t}%
\end{bmatrix}%
=\left[ 
\begin{array}{cc}
b_{11} & b_{12} \\ 
b_{21} & b_{22}%
\end{array}%
\right] 
\begin{bmatrix}
x_{1t-1} \\ 
x_{2t-1}%
\end{bmatrix}%
+%
\begin{bmatrix}
\varepsilon _{1} \\ 
\varepsilon _{2}%
\end{bmatrix}%
.
\end{equation*}%
Or in other words that:

\begin{itemize}
\item 
\begin{itemize}
\item $\varepsilon _{\Delta yt}$ affects contemporaneously all variables,
namely $\Delta y_{t}$, $\pi _{t}$ and $r_{t}$

\item $\varepsilon _{\pi t}$ affects contemporaneously only $\pi _{t}$ and $%
r_{t}$, but no $\Delta y_{t}$

\item $\varepsilon _{rt}$ affects contemporaneously only $r_{t}$
\end{itemize}
\end{itemize}

\noindent We now have 3 unknowns and 3 equations! To see what are the
implications of our assumptions on $\mathbf{A}$, first remember that the
inverse of a lower triangular matrix is also lower triangular%
\begin{equation*}
\left[ 
\begin{array}{cc}
a_{11} & {%
%TCIMACRO{\TeXButton{note}{\color{red}}}%
%BeginExpansion
\color{red}%
%EndExpansion
{0}} \\ 
a_{21} & a_{22}%
\end{array}%
\right] ^{-1}=\left[ 
\begin{array}{cc}
\tilde{a}_{11} & {%
%TCIMACRO{\TeXButton{note}{\color{red}}}%
%BeginExpansion
\color{red}%
%EndExpansion
{0}} \\ 
\tilde{a}_{21} & \tilde{a}_{22}%
\end{array}%
\right]
\end{equation*}%
Now pre-multiply the VAR\ by $\mathbf{A}^{-1}$%
\begin{equation*}
\begin{bmatrix}
x_{1t} \\ 
x_{2t}%
\end{bmatrix}%
=\left[ 
\begin{array}{cc}
f_{11} & f_{12} \\ 
f_{21} & f_{22}%
\end{array}%
\right] 
\begin{bmatrix}
x_{1t-1} \\ 
x_{2t-1}%
\end{bmatrix}%
+\left[ 
\begin{array}{cc}
_{11} & {%
%TCIMACRO{\TeXButton{note}{\color{red}}}%
%BeginExpansion
\color{red}%
%EndExpansion
{0}} \\ 
\tilde{a}_{21} & \tilde{a}_{22}%
\end{array}%
\right]%
\begin{bmatrix}
\varepsilon _{1} \\ 
\varepsilon _{2}%
\end{bmatrix}%
\end{equation*}%
Which implies that 
\begin{equation*}
\left\{ 
\begin{array}{ll}
x_{1t} & =...+\tilde{a}_{11}\varepsilon _{1}\smallskip \\ 
x_{2t} & =...+\tilde{a}_{21}\varepsilon _{1}+\tilde{a}_{22}\varepsilon _{2}%
\end{array}%
\right.
\end{equation*}%
We normally implement this identification scheme \emph{via} a {Cholesky
decomposition} of $\Sigma _{\mathbf{u}}$:%
\begin{equation*}
\Sigma _{\mathbf{u}}=\mathbf{P}^{\prime }\mathbf{P,}
\end{equation*}%
where $\mathbf{P}^{\prime }$ is lower triangular. Note that%
\begin{equation*}
\Sigma _{\mathbf{u}}=\mathbf{P}^{\prime }\mathbf{P}\text{ \ \ \ \ but also \
\ \ \ }\Sigma _{\mathbf{u}}=\mathbf{A}^{-1}\mathbf{A}^{-1\prime }
\end{equation*}%
and that $\mathbf{A}^{-1}$ is lower triangular as $\mathbf{P}^{\prime }$.
Then it musty follow that $\mathbf{P}^{\prime }\mathbf{=A}%
^{-1}\Longrightarrow $ Identification!

%TCIMACRO{%
%\TeXButton{---Note}{\begin{quotation}
%\noindent \rule{1cm}{.4pt}   {\scshape Note}   \hrulefill
%\vspace{-.36cm}
%
%\noindent \hrulefill
%\small
%\singlespacing
%\color{note}\sffamily}}%
%BeginExpansion
\begin{quotation}
\noindent \rule{1cm}{.4pt}   {\scshape Note}   \hrulefill
\vspace{-.36cm}

\noindent \hrulefill
\small
\singlespacing
\color{note}\sffamily%
%EndExpansion

Don't be scared of Cholesky decomposition! It's a kind of square root of a
matrix. As in Excel you type \texttt{sqrt()} in Matlab you type \texttt{%
chol()}. A symmetric and positive definite matrix $\mathbf{X}$ can be
decomposed as:%
\begin{equation*}
\mathbf{X}=\mathbf{P}^{\prime }\mathbf{P,}
\end{equation*}%
where $\mathbf{P}$ is an upper triangular matrix (and therefore $\mathbf{P}%
^{\prime }$ is lower triangular). The formula is 
\begin{equation*}
\mathbf{X}=%
\begin{bmatrix}
a & b \\ 
b & c%
\end{bmatrix}%
\ \ \ \mathbf{P}=%
\begin{bmatrix}
\sqrt{a} & \frac{b}{\sqrt{a}} \\ 
0 & \sqrt{c-\frac{b^{2}}{a}}%
\end{bmatrix}%
\end{equation*}

%TCIMACRO{%
%\TeXButton{---Note End}{\color{black}
%\noindent \hrulefill 
%\vspace{-.5cm}
%
%\noindent \hrulefill
%\end{quotation}
%\bigskip}}%
%BeginExpansion
\color{black}
\noindent \hrulefill 
\vspace{-.5cm}

\noindent \hrulefill
\end{quotation}
\bigskip%
%EndExpansion

\subsection{Zero short-run restrictions in Matlab}

The computation of the $\mathbf{A}^{-1}$ matrix in Matlab is
straightforward. Consider the following \texttt{DATA} matrix (as the one
considered above):%
\begin{equation*}
\mathtt{DATA}=\mathbf{x}^{\prime }=\left[ 
\begin{array}{cc}
x_{11} & x_{21} \\ 
x_{12} & x_{22} \\ 
... & ... \\ 
x_{1T} & x_{2T}%
\end{array}%
\right] .
\end{equation*}%
and a bivariate VAR(1) with no constant and no trend (for
simplicity):\medskip

\texttt{VARout = VARmodel(Y,0,1);}\medskip

\noindent The results are stored in the \texttt{VARout} structure as follows:%
\begin{eqnarray*}
\mathtt{VARout.beta} &=&\mathbf{F}^{\prime }=\left[ 
\begin{array}{cc}
f_{11} & f_{21} \\ 
f_{12} & f_{22}%
\end{array}%
\right] , \\
\mathtt{VARout.sigma} &=&\mathbf{\Sigma }_{u}=%
\begin{bmatrix}
\sigma _{1}^{2} & \sigma _{12}^{2} \\ 
- & \sigma _{2}^{2}%
\end{bmatrix}%
.
\end{eqnarray*}%
First, let's transform the matrix of estimated coefficients to be consistent
with the notation above, namely:\medskip

\texttt{F = transpose(VARout.beta);}\medskip

\noindent where \texttt{F }$\mathbf{=F\ }$and rename the VCV\ matrix of the
reduced form residuals:\medskip

\texttt{sigma = VARout.sigma;}\medskip

\noindent where \texttt{sigma }$=\mathbf{\Sigma }_{u}$. The $\mathbf{A}^{-1}$
matrix can now be computed with the following line of code:\medskip

\texttt{invA = chol(sigma)';}\medskip

\noindent where \texttt{invA }$=\mathbf{A}^{-1}$ is a lower triangular
matrix.

\section{Structural VARs}

Reduced form VAR representations, such as those introduced by \cite{Sims1980}%
, do not allow for instantaneous relationship among the dependent variables.
As a result, there is a correlation structure among the error term that is
left undetermined. The zero short-run restrictions identification scheme
described above allows to decouple the correlation structure in the
reduced-form residuals and to recover the structural errors.

The recursive structure implied by the zero short-run restrictions
identification scheme, however, may not always be sensible. There have been
many different types of restriction suggested in order to achieve the
identification of the structural shocks. A summary of different types of
restriction can be found in \cite{ChristianoEichenbaumEvans1999}.

\section{Zero long-run restrictions}

This section will focus on identification based on long-run restrictions. As
an example of long-run restrictions, let us consider the \cite%
{BlanchardQuah1989} identification of demand and supply shocks. Economic
theory usually tells us a lot more about what will happen in the long-run,
rather than exactly what will happen today. For instance, theory tells us
that whatever positive aggregate demand shocks do in the short-run, in the
long-run they have no effect on output and a positive effect on the price
level. This suggests an alternative approach: to use these
theoretically-inspired long-run restrictions to identify shocks and impulse
responses.

Consider a reduced form bivariate VAR as in equation \ref{eq:var_reduced_c}
and its structural representation:%
\begin{eqnarray*}
\mathbf{x}_{t} &=&\mathbf{Fx}_{t-1}+\mathbf{u}_{t} \\
\mathbf{Ax}_{t} &=&\mathbf{Bx}_{t-1}+\mathbf{\varepsilon }_{t}.
\end{eqnarray*}%
To simplify the notation, as we have done above we define $\mathbf{A}%
^{-1}\equiv \mathbf{\tilde{A}}$. We can therefore re-write the VAR as:%
\begin{equation*}
\mathbf{x}_{t}=\mathbf{\Phi x}_{t-1}+\mathbf{\tilde{A}\varepsilon }_{t},
\end{equation*}%
and express the VCV matrix of the residuals of the reduced form VAR as:%
\begin{equation}
\mathbf{\Sigma }_{u}=\mathsf{E}\left[ \mathbf{u}_{t}\mathbf{u}_{t}^{\prime }%
\right] =\mathsf{E}\left[ \mathbf{\tilde{A}\varepsilon }_{t}\mathbf{%
\varepsilon }^{\prime }\mathbf{\tilde{A}}^{\prime }\right] =\mathbf{\tilde{A}%
\Sigma }_{\varepsilon }\mathbf{\tilde{A}}^{\prime }=\mathbf{\tilde{A}\tilde{A%
}}^{\prime }.  \label{eq:sigma_u_BQ}
\end{equation}

If a structural shock hits in time $t$, its long run impact on the level of $%
\mathbf{x}_{t}$ ---given that the VAR is stable, i.e. that is if eigenvalues
of $\mathbf{F}$ are are inside unit circle--- is given by:%
\begin{equation}
\begin{array}{l}
\mathbf{x}_{t+\infty }=\mathbf{\tilde{A}\varepsilon }_{t}\mathbf{+F\tilde{A}%
+F}^{2}\mathbf{\tilde{A}\varepsilon }_{t}+...+\mathbf{F}^{\infty }\mathbf{%
\tilde{A}\varepsilon }_{t}, \\ 
\mathbf{x}_{t}=\left( \mathbf{I}-\mathbf{F}\right) ^{-1}\mathbf{\tilde{A}%
\varepsilon }_{t},%
\end{array}
\label{eq:VMA_red}
\end{equation}%
where remember that the series $\mathbf{I}+\mathbf{F}+\mathbf{F}^{2}+...%
\mathbf{F}^{\infty }=\left( \mathbf{I}-\mathbf{F}\right) ^{-1}$ if the
eigenvalues of $\mathbf{F}$ are smaller than one. Moreover we can simplify
the above expression by defining: 
\begin{equation}
\mathbf{x}_{t}=\mathbf{D\varepsilon }_{t},  \label{eq:VMA_struct}
\end{equation}%
where $\mathbf{D}=\left( \mathbf{I}-\mathbf{F}\right) ^{-1}\mathbf{\tilde{A}}
$. The matrix $\mathbf{D}$ represents the cumulative effect (from $t$ to $%
t+\infty $) of a shock hitting in period $t$.

The idea of Blanchard and Quah is the following. According to the natural
rate hypothesis, demand-side shocks will have no long-run effect on the
level of output, while supply-side shocks (e.g., productivity shocks) will
have a permanent effect on it. Assume that $\varepsilon _{1t}$ is the supply
side shock and that $\varepsilon _{2t}$ is the demand side shock: we can
rewrite (%
\index{eq:VMA\_struct}) such that the cumulated effect of $\varepsilon _{2t}$
on $x_{1t}$ is equal to zero by assuming:%
\begin{equation*}
\left[ 
\begin{array}{c}
x_{1t} \\ 
x_{2t}%
\end{array}%
\right] =\left[ 
\begin{array}{cc}
d_{11} & 0 \\ 
d_{21} & d_{22}%
\end{array}%
\right] \left[ 
\begin{array}{c}
\varepsilon _{1t} \\ 
\varepsilon _{2t}%
\end{array}%
\right] ,
\end{equation*}%
by imposing $d_{12}=0$.

How can we compute the value of $\mathbf{D}$? First let's compute $\mathbf{DD%
}^{\prime }$:%
\begin{equation*}
\mathbf{DD}^{\prime }=\left( \mathbf{I}-\mathbf{F}\right) ^{-1}\mathbf{%
\tilde{A}\tilde{A}}^{\prime }\left( \left( \mathbf{I}-\mathbf{F}\right)
^{-1}\right) ^{\prime },
\end{equation*}%
\newline
and then remember from equation (\ref{eq:sigma_u_BQ}) that $\mathbf{\tilde{A}%
\tilde{A}}^{\prime }=\mathbf{\Sigma }_{u}$. Therefore, all elements of that
the right hand side of the previous equation are known and $\mathbf{DD}%
^{\prime }$ is pinned down:%
\begin{equation*}
\mathbf{DD}^{\prime }=\left( \mathbf{I}-\mathbf{F}\right) ^{-1}\mathbf{%
\Sigma }_{u}\left( \left( \mathbf{I}-\mathbf{F}\right) ^{-1}\right) ^{\prime
}.
\end{equation*}%
Now that we know $\mathbf{DD}^{\prime }$, how can we pin down $\mathbf{D}$?
To solve for $\mathbf{D}$, remember that both $\mathbf{DD}^{\prime }$ and $%
\left( \mathbf{I}-\mathbf{F}\right) ^{-1}\mathbf{\Sigma }_{u}\left( \left( 
\mathbf{I}-\mathbf{F}\right) ^{-1}\right) ^{\prime }$ are symmetric matrices.

All symmetric matrices have a unique upper diagonal matrix $\mathbf{P}$ such
that $\mathbf{P}^{\prime }\mathbf{P}$ equals the original symmetric matrix
(the Cholesky factor of the symmetric matrix). If we take the Cholesky
decomposition of $\left( \mathbf{I}-\mathbf{F}\right) ^{-1}\mathbf{\Sigma }%
_{u}\left( \left( \mathbf{I}-\mathbf{F}\right) ^{-1}\right) ^{\prime }$ such
that: 
\begin{equation*}
\mathbf{P}^{\prime }\mathbf{P=}\left( \mathbf{I}-\mathbf{F}\right) ^{-1}%
\mathbf{\Sigma }_{u}\left( \left( \mathbf{I}-\mathbf{F}\right) ^{-1}\right)
^{\prime },
\end{equation*}%
and we assume that $d_{12}=0$ (as we did above), $\mathbf{D}$ is uniquely
pinned down by:%
\begin{equation*}
\mathbf{D}=\left[ 
\begin{array}{cc}
d_{11} & 0 \\ 
d_{21} & d_{22}%
\end{array}%
\right] =\mathbf{P}^{\prime }
\end{equation*}%
Now that we know $\mathbf{D}$, we can easily recover the structural matrix $%
\mathbf{A}^{-1}=\mathbf{\tilde{A}}=\left( \mathbf{I}-\mathbf{F}\right) 
\mathbf{D}$.

%TCIMACRO{%
%\TeXButton{---Note}{\begin{quotation}
%\noindent \rule{1cm}{.4pt}   {\scshape Note}   \hrulefill
%\vspace{-.36cm}
%
%\noindent \hrulefill
%\small
%\singlespacing
%\color{note}\sffamily}}%
%BeginExpansion
\begin{quotation}
\noindent \rule{1cm}{.4pt}   {\scshape Note}   \hrulefill
\vspace{-.36cm}

\noindent \hrulefill
\small
\singlespacing
\color{note}\sffamily%
%EndExpansion

The computation of the $\mathbf{D}$ matrix in Matlab is straightforward.
Consider a bivariate VAR(1) with no constant and no trend (as the one
considered above):\medskip

\texttt{VARout = VARmodel(Y,0,1);}

\texttt{F = transpose(VARout.beta);}

\texttt{sigma = VARout.sigma}\medskip ;

\noindent where \texttt{F }$\mathbf{=F}$ and \texttt{sigma }$=\mathbf{\Sigma 
}_{u}$. The $\mathbf{D}$ matrix can now be computed with the following lines
of code:\medskip

\texttt{Finf = inv(eye(length(F))-F);}

\texttt{D = chol(Finf*sigma*Finf')';}\medskip

\noindent where \texttt{Finf }$\mathbf{=\left( \mathbf{I}-\mathbf{F}\right)
^{-1}}$ and \texttt{D }$\mathbf{=D}$. Finally, the structural matrix $%
\mathbf{A}^{-1}$ can also be computed as\medskip

\texttt{invA = Finf\TEXTsymbol{\backslash}D;}\medskip

\noindent where \texttt{invA }$=\mathbf{\tilde{A}}=\mathbf{A}^{-1}$.

%TCIMACRO{%
%\TeXButton{---Note End}{\color{black}
%\noindent \hrulefill 
%\vspace{-.5cm}
%
%\noindent \hrulefill
%\end{quotation}
%\bigskip}}%
%BeginExpansion
\color{black}
\noindent \hrulefill 
\vspace{-.5cm}

\noindent \hrulefill
\end{quotation}
\bigskip%
%EndExpansion

Therefore, we just computed a matrix $\mathbf{D}$, such that $\varepsilon
_{2t}$ has no long-run effect on $x_{1t}$. In their original paper,
Blanchard and Quah used a bivariate VAR in the log-difference of GDP ($%
\Delta y_{t}$) and the level of unemployment rate ($u_{t}$). The
lower-diagonal assumption thus meant that of the two structural shocks only $%
\varepsilon _{1t}$ ---what they label a supply shock--- could have a
long-run effect on the level of output (i.e., the cumulated response of $%
\Delta y_{t}$).

\section{Sign restrictions}

In the zero short-run restriction identification we used the fact that%
\begin{equation*}
\mathbf{\hat{\Sigma}}_{u}=\mathbf{A}^{-1}\mathbf{\Sigma }_{\varepsilon
}\left( \mathbf{A}^{-1}\right) ^{\prime }\text{ \ \ \ and \ \ \ }\mathbf{%
\hat{\Sigma}}_{u}=\mathbf{P}^{\prime }\mathbf{P,}
\end{equation*}%
where the lower triangular $\mathbf{P}^{\prime }$ matrix is the Cholesky
decomposition of the variance-covariance matrix of the reduced-form error
terms. Remember that Identification was achieved by assuming that $\mathbf{A}%
^{-1}$ was also lower triangular.

Note however that such decomposition is not unique. Specifically, for a
given orthonormal matrix $\mathbf{S}$ such that:%
\begin{equation*}
\mathbf{S}^{\prime }\mathbf{S}=\mathbf{I},
\end{equation*}%
we have that:%
\begin{equation*}
\mathbf{\hat{\Sigma}}_{u}=\mathbf{A}^{-1}\mathbf{\Sigma }_{\varepsilon
}\left( \mathbf{A}^{-1}\right) ^{\prime }=\mathbf{P}^{\prime }\mathbf{S}%
^{\prime }\mathbf{SP=}\mathcal{P}^{\prime }\mathcal{P}\mathbf{.}
\end{equation*}%
where $\mathcal{P}^{\prime }$ is generally not lower triangular anymore.
Therefore, the system of equations implied by $\mathbf{A}^{-1}\left( \mathbf{%
A}^{-1}\right) ^{\prime }=\mathcal{P}^{\prime }\mathcal{P}$ can be solved
(there are as many equations as the number of unknowns!): but is the
solution plausible? Identification is achieved by checking whether the
impulse response function (formally defined in the section below) implied by
this solution satisfy a set of a priori (and possibly theory-driven) sign
restrictions.

Specifically, the sign restriction identification procedure can be
summarized by the following steps:

\begin{enumerate}
\item Draw a random orthonormal matrix $\mathbf{S}^{\prime }$

\item Compute the implied $\mathbf{A}^{-1}=\mathbf{P}^{\prime }\mathbf{S}%
^{\prime }$ where $\mathbf{P}^{\prime }$ is the Cholesky decomposition of $%
\mathbf{\hat{\Sigma}}_{u}$

\item Compute the impulse response associated with $\mathbf{A}^{-1}$. Are
the sign restrictions satisfied?

\begin{enumerate}
\item Yes. Store the impulse response

\item No. Discard the impulse response
\end{enumerate}

\item Perform $N\ $replications and report the median impulse response (and
its confidence intervals)
\end{enumerate}

%TCIMACRO{%
%\TeXButton{---Note}{\begin{quotation}
%\noindent \rule{1cm}{.4pt}   {\scshape Note}   \hrulefill
%\vspace{-.36cm}
%
%\noindent \hrulefill
%\small
%\singlespacing
%\color{note}\sffamily}}%
%BeginExpansion
\begin{quotation}
\noindent \rule{1cm}{.4pt}   {\scshape Note}   \hrulefill
\vspace{-.36cm}

\noindent \hrulefill
\small
\singlespacing
\color{note}\sffamily%
%EndExpansion

In Matlab we can easily implement the sign restriction identification
scheme. Consider a bivariate VAR(1) with no constant and no trend (as the
one considered above):\medskip

\texttt{VARout = VARmodel(Y,0,1);}

\texttt{sigma = VARout.sigma}\medskip ;

\noindent where \texttt{sigma }$=\mathbf{\Sigma }_{u}$. For each draw we can
compute $\mathbf{A}^{-1}=\mathbf{P}^{\prime }\mathbf{S}^{\prime }$ as
follows. First we draw a random orthonormal matrix with the \texttt{%
OrthNorm.m} function\medskip

\texttt{S = OrthNorm(length(sigma));}\medskip

\noindent where \texttt{S }$=\mathbf{S.}$ Then we compute\medskip

\texttt{invA = chol(sigma)'*S';}\medskip

\noindent where \texttt{invA }$=\mathbf{A}^{-1}=\mathbf{P}^{\prime }\mathbf{S%
}^{\prime }$.

%TCIMACRO{%
%\TeXButton{---Note End}{\color{black}
%\noindent \hrulefill 
%\vspace{-.5cm}
%
%\noindent \hrulefill
%\end{quotation}
%\bigskip}}%
%BeginExpansion
\color{black}
\noindent \hrulefill 
\vspace{-.5cm}

\noindent \hrulefill
\end{quotation}
\bigskip%
%EndExpansion

\chapter{IMPULSE RESPONSE FUNCTIONS}

In a VAR we are often interested in obtaining the impulse response
functions. Impulse responses trace out the response of current and future
values of each of the variables to a one-unit increase (or to a one-standard
deviation increase, when the scale matters) in the current value of one of
the VAR errors, assuming that this error returns to zero in subsequent
periods and that all other errors are equal to zero.

The implied thought experiment of changing one error while holding the
others constant makes most sense when the errors are uncorrelated across
equations, so impulse responses are typically calculated for recursive and
structural VARs. On the contrary, the estimated error terms $u_{1t}$ and $%
u_{2t}$ are correlated: it doesn't make much sense to ask \textquotedblleft
what if $u_{1t}$ has a unit impulse with no change in $u_{2t}$", because
when I am moving $u_{1t}$, I am moving the structural shock of both
equations (remember that the $u$'s are a linear combination of the
structural shocks).

\section{IRF and the moving average representation}

Consider the general structural form of a bivariate VAR(1), $\mathbf{Ax}_{t}=%
\mathbf{Bx}_{t-1}+\mathbf{\varepsilon }_{t}$, and its reduced form \textbf{$%
x $}$_{t}=\mathbf{Fx}_{t-1}+\mathbf{u}_{t}$. Using the lag operator, we can
re-write the same model as:%
\begin{equation*}
\left( \mathbf{I}-\mathbf{F}L\right) \mathbf{x}_{t}=\mathbf{u}_{t}.
\end{equation*}%
If the roots of $\mathbf{I}-\mathbf{F}L$ lie outside the unit circle, than
the time series is stationary and we can write:%
\begin{equation*}
\mathbf{x}_{t}=\sum_{i=0}^{\infty }\mathbf{F}^{i}\mathbf{u}%
_{t-i}=\sum_{i=0}^{\infty }\left[ 
\begin{array}{cc}
f_{11} & f_{12} \\ 
f_{21} & f_{22}%
\end{array}%
\right] ^{i}\left[ 
\begin{array}{c}
u_{1t-i} \\ 
u_{2t-i}%
\end{array}%
\right] .
\end{equation*}%
Moreover, since we know that $\mathbf{\tilde{A}\varepsilon }$, we can write
the above expression as a function of the structural errors:%
\begin{equation*}
\mathbf{x}_{t}=\sum_{i=0}^{\infty }\left[ 
\begin{array}{cc}
f_{11} & f_{12} \\ 
f_{21} & f_{22}%
\end{array}%
\right] ^{i}%
\begin{bmatrix}
\tilde{a}_{11} & \tilde{a}_{12} \\ 
\tilde{a}_{21} & \tilde{a}_{22}%
\end{bmatrix}%
\left[ 
\begin{array}{c}
\varepsilon _{1t} \\ 
\varepsilon _{2t}%
\end{array}%
\right] ,
\end{equation*}%
or, by defining $\mathbf{F}^{i}\mathbf{\tilde{A}}=\mathbf{\Gamma }^{i}$: 
\begin{equation*}
\begin{bmatrix}
x_{1t} \\ 
x_{2t}%
\end{bmatrix}%
=\sum_{i=0}^{\infty }%
\begin{bmatrix}
\gamma _{11} & \gamma _{22} \\ 
\gamma _{12} & \gamma _{22}%
\end{bmatrix}%
^{i}\left[ 
\begin{array}{c}
\varepsilon _{1t} \\ 
\varepsilon _{2t}%
\end{array}%
\right] ,
\end{equation*}%
which in terms of a system of linear equations is given by:%
\begin{eqnarray*}
x_{1t} &=&\varepsilon _{1t}+[\gamma _{11}^{1}\varepsilon _{1,t-1}+\gamma
_{12}^{1}\varepsilon _{2,t-1}]+[\gamma _{11}^{2}\varepsilon _{1,t-2}+\gamma
_{12}^{2}\varepsilon _{2,t-2}]+... \\
x_{2t} &=&\varepsilon _{2t}+[\gamma _{21}^{1}\varepsilon _{1,t-1}+\gamma
_{22}^{1}\varepsilon _{2,t-1}]+[\gamma _{21}^{2}\varepsilon _{1,t-2}+\gamma
_{22}^{2}\varepsilon _{2,t-2}]+...
\end{eqnarray*}%
Now suppose we are interested in the response of $x_{1t}$ to a shock in the
structural error of equation 1, $\varepsilon _{1t}$, from period $t$ to
period $t+h$. First, let's forward the system by $h$ periods:%
\begin{eqnarray*}
x_{1t+h} &=&\varepsilon _{1t+h}+[\gamma _{11}^{1}\varepsilon
_{1,t+h-1}+\gamma _{12}^{1}\varepsilon _{2,t+h-1}]+...+[\gamma
_{11}^{h}\varepsilon _{1,t}+\gamma _{12}^{h}\varepsilon _{2,t}]+... \\
x_{2t+h} &=&\varepsilon _{2t+h}+[\gamma _{21}^{1}\varepsilon
_{1,t+h-1}+\gamma _{22}^{1}\varepsilon _{2,t+h-1}]+...+[\gamma
_{21}^{h}\varepsilon _{1,t}+\gamma _{22}^{h}\varepsilon _{2,t}]+...
\end{eqnarray*}%
Then, we can compute the effect we are interested in as:%
\begin{equation*}
\frac{\partial x_{1,t+h}}{\partial \varepsilon _{1,t}}=\gamma _{11}^{h}.
\end{equation*}%
The Impulse Response Function is defined as $IRF=\Gamma ^{h}$ for $%
n=0,1,...H $. The elements of the matrix $\Gamma ^{h}$ are called \textbf{%
impact multipliers} while the \textbf{long-run cumulated effect} is given by 
$\sum_{i=1}^{\infty }\Gamma ^{i}$.

\section{How to compute IRFs in practice?}

Let's define $\mathbf{s}_{t}=\left( s_{1t}^{\prime },s_{2t}^{\prime }\right)
^{\prime }$ as a vector of exogenous impulses that we want to impose to the
structural errors of the system. Impulse responses allow us to simulate the
effect of $\mathbf{s}_{t}$ on the dynamics of our VAR\ model.

For example, suppose we are interested in a one standard deviation impulse
to the structural error of equation of $x_{1}$ that returns to zero in
subsequent periods. Therefore, the time profile of the impulse will be:%
\begin{equation*}
\begin{tabular}{lllll}
\hline
$\text{Time (}\tau \text{)}$ &  & $1$ & $2$ & $h$ \\ \hline
$\text{Impulse to }\varepsilon _{1t}\text{ (}s_{1}\text{)}$ &  & $s_{1,1}=1$
& $s_{1,2}=0$ & $s_{1,h}=0$ \\ 
$\text{Impulse to }\varepsilon _{2t}\text{ (}s_{2}\text{)}$ &  & $s_{2,1}=0$
& $s_{2,2}=0$ & $s_{2,h}=0$ \\ \hline
\end{tabular}%
.
\end{equation*}%
After identifying the VAR\ with one of the methodologies described above, we
can use the reduced-form representation of our VAR\ to compute the IRFs.
Specifically, ewe can write%
\begin{equation*}
\mathbf{x}_{t}=\mathbf{Fx}_{t-1}+\mathbf{A}^{-1}\mathbf{s}_{t},
\end{equation*}%
which implies that the impulse response $IRF\tau $ in time $\tau =1$ is:%
\begin{eqnarray*}
\mathbf{IRF}_{1} &=&\mathbf{A}^{-1}\mathbf{s}_{1}, \\
\mathbf{IRF}_{\tau } &=&\mathbf{FIRF}_{\tau -1},\text{ \ \ \ \ for }\tau
=2,...h.
\end{eqnarray*}

\chapter{HISTORICAL DECOMPOSITIONS}

Take a VAR with trend and constant. Each observation can be written as%
\begin{equation*}
\begin{array}{lll}
\mathbf{x}_{1} & = & \mathbf{c}+1+\mathbf{Fy}_{0}+\mathbf{u}_{1} \\ 
\mathbf{x}_{2} & = & \mathbf{c}+2+\mathbf{Fy}_{1}+\mathbf{u}_{2}\mathbf{=%
\mathbf{c}}+2+\mathbf{F(\mathbf{c}}+\mathbf{1}+\mathbf{\mathbf{Fy}_{0}}+%
\mathbf{\mathbf{u}_{1})+u}_{2}=\mathbf{F}^{2}\mathbf{x}_{0}+\underset{cc}{%
\underbrace{(\mathbf{Fc}+\mathbf{c)}}}+(\mathbf{F}+\mathbf{2)}+\underset{uu}{%
\underbrace{(\mathbf{Fu}_{1}+\mathbf{u}_{2})}} \\ 
\mathbf{x}_{3} & = & \mathbf{c}+3+\mathbf{Fy}_{2}+\mathbf{u}_{3}\mathbf{=F}%
^{3}\mathbf{x}_{0}+(\mathbf{F}^{2}\mathbf{c}+\mathbf{Fc}+\mathbf{c)}+(%
\mathbf{F}^{2}+2\mathbf{F}+\mathbf{3)}+(\mathbf{F}^{2}\mathbf{u}_{1}+\mathbf{%
Fu}_{2}+\mathbf{u}_{3}) \\ 
& = & ...\ = \\ 
\mathbf{x}_{T} & = & \mathbf{c}+T+\mathbf{Fy}_{T-1}+\mathbf{u}_{T}\mathbf{=F}%
^{T}\mathbf{x}_{0}+(\mathbf{F}^{T}\mathbf{c}+...+\mathbf{Fc}+\mathbf{c)}+(%
\mathbf{F}^{T}+...+\mathbf{F(}T-1)+T\mathbf{)}+(\mathbf{F}^{T}\mathbf{u}%
_{0}+...+\mathbf{Fu}_{T-1}+\mathbf{u}_{T})%
\end{array}%
\end{equation*}%
Now take structural errors%
\begin{equation*}
\begin{array}{lll}
\mathbf{x}_{1} & = & \mathbf{Fy}_{0}+\mathbf{A}^{-1}\mathbf{\varepsilon }_{1}
\\ 
\mathbf{x}_{2} & = & \mathbf{Fy}_{1}+\mathbf{A}^{-1}\mathbf{\varepsilon }_{2}%
\mathbf{=F(\mathbf{Fy}_{0}}+\mathbf{A}^{-1}\mathbf{\varepsilon }_{1}\mathbf{%
)+A}^{-1}\mathbf{\varepsilon }_{2}=\mathbf{F}^{2}\mathbf{x}_{0}+\underset{uu}%
{\underbrace{(\mathbf{FA}^{-1}\mathbf{\varepsilon }_{1}+\mathbf{A}^{-1}%
\mathbf{\varepsilon }_{2})}} \\ 
\mathbf{x}_{3} & = & \mathbf{c}+3+\mathbf{Fy}_{2}+\mathbf{u}_{3}\mathbf{=F}%
^{3}\mathbf{x}_{0}+(\mathbf{F}^{2}\mathbf{c}+\mathbf{Fc}+\mathbf{c)}+(%
\mathbf{F}^{2}+2\mathbf{F}+\mathbf{3)}+(\mathbf{F}^{2}\mathbf{u}_{1}+\mathbf{%
Fu}_{2}+\mathbf{u}_{3}) \\ 
& = & ...\ = \\ 
\mathbf{x}_{T} & = & \mathbf{c}+T+\mathbf{Fy}_{T-1}+\mathbf{u}_{T}\mathbf{=F}%
^{T}\mathbf{x}_{0}+(\mathbf{F}^{T}\mathbf{c}+...+\mathbf{Fc}+\mathbf{c)}+(%
\mathbf{F}^{T}+...+\mathbf{F(}T-1)+T\mathbf{)}+(\mathbf{F}^{T}\mathbf{u}%
_{0}+...+\mathbf{Fu}_{T-1}+\mathbf{u}_{T})%
\end{array}%
\end{equation*}

\singlespacing{\small 
\bibliographystyle{econometrica}
\bibliography{References_VAR}
}

\end{document}
