%\newcommand{\todo}[1]{\todo[color=gray!20,inline]{\textbf{Note}: #1}}
%\makeatletter
%\renewcommand\verbatim@font{\footnotesize\footnotesize\ttfamily}
%\makeatother


\documentclass[10pt]{article}
%%%%%%%%%%%%%%%%%%%%%%%%%%%%%%%%%%%%%%%%%%%%%%%%%%%%%%%%%%%%%%%%%%%%%%%%%%%%%%%%%%%%%%%%%%%%%%%%%%%%%%%%%%%%%%%%%%%%%%%%%%%%%%%%%%%%%%%%%%%%%%%%%%%%%%%%%%%%%%%%%%%%%%%%%%%%%%%%%%%%%%%%%%%%%%%%%%%%%%%%%%%%%%%%%%%%%%%%%%%%%%%%%%%%%%%%%%%%%%%%%%%%%%%%%%%%
\usepackage{eurosym}
\usepackage{amsfonts}
\usepackage{amsmath}
\usepackage{booktabs}
\usepackage[a4paper,hmargin={1.4in,1.4in},vmargin={1.4in,1.4in}]{geometry}
\usepackage{mathpazo}
\usepackage[scaled]{helvet}
\usepackage{courier}
\usepackage[T1]{fontenc}
\usepackage{verbatim}
\usepackage{color}
\usepackage{graphicx}
\usepackage{sectsty}
\usepackage{enumitem}
\usepackage{fancyvrb}
\usepackage{newverbs}
\usepackage[scaled=.8]{nimbusmono}
\usepackage[bordercolor=white,backgroundcolor=gray!30,linecolor=black,colorinlistoftodos]{todonotes}
\usepackage[square,sort,comma,numbers]{natbib}\usepackage[bookmarks=true,pdfauthor=A.Cesa-Bianchi,colorlinks=true,linkcolor=red,citecolor=note,urlcolor=note]{hyperref}

\setcounter{MaxMatrixCols}{10}
%TCIDATA{OutputFilter=Latex.dll}
%TCIDATA{Version=5.50.0.2953}
%TCIDATA{<META NAME="SaveForMode" CONTENT="1">}
%TCIDATA{BibliographyScheme=Manual}
%TCIDATA{LastRevised=Sunday, March 15, 2020 19:48:57}
%TCIDATA{<META NAME="GraphicsSave" CONTENT="32">}
%TCIDATA{Language=American English}

\newlength\myverbindent 
\setlength\myverbindent{1cm} \makeatletter
\def\verbatim@processline{  \hspace{\myverbindent}\the\verbatim@line\par}
\makeatother
\definecolor{cmd}{gray}{0.99}
\definecolor{script}{RGB}{255,248,225}
\linespread{1} 
\definecolor{subsection}{RGB}{0,0,50}
\definecolor{section}{RGB}{0,0,50}
\definecolor{blue}{RGB}{0,0,100}
\sectionfont{\color{section}}
\subsectionfont{\color{subsection}}
\newenvironment{proof}[1][Proof]{\noindent\textbf{#1.} }{\ \rule{0.5em}{0.5em}}
\renewcommand*\familydefault{\sfdefault}
\graphicspath{{./graphics/}}
\newcommand*\backmatter{\setcounter{section}{0}\renewcommand\theHsection{back.\Roman{section}}}
\definecolor{shadecolor}{rgb}{0.95,0.95,0.95}
\definecolor{title}{RGB}{0,0,90}
\definecolor{note}{RGB}{49,79,179}
\definecolor{light}{RGB}{70,130,180}
\definecolor{red}{RGB}{200,0,0}
\definecolor{gold}{RGB}{218,165,32}
\definecolor{green}{RGB}{0,179,0}
\definecolor{purple}{RGB}{150,40,160}
\definecolor{maroon}{RGB}{128,0,0}
\definecolor{red}{RGB}{140,0,0}
\definecolor{blue}{RGB}{0,0,255}
\definecolor{maroon}{RGB}{128,0,0}
\definecolor{subsection}{RGB}{0,0,90}
\definecolor{section}{RGB}{0,0,90}
\definecolor{matlabgreen}{RGB}{0,153,0}
\definecolor{matlabpurple}{RGB}{127,0,255}
\definecolor{matlabblue}{RGB}{0,42,252}
\sectionfont{\color{section}}
\subsectionfont{\color{subsection}}
\setlength{\parskip}{.25cm}
\setlength{\parindent}{0cm}
\setlist[itemize]{topsep=0cm}
\setlist[itemize]{topsep=0cm}
\renewcommand*\familydefault{\sfdefault}
\renewcommand\labelitemi{-}
\setcounter{secnumdepth}{3}
\graphicspath{{./graphics/}}
\input{tcilatex}
\begin{document}


\section{Installing the VAR Toolbox}

No installation is required. Simply extract the codes from the\ ZIP\ file
and copy them to a specific folder, e.g. \textquotedblleft \texttt{%
C:/UserFolder/VARToolbox}\textquotedblright . Then, add the folder (with
subfolders)\ to the Matlab path. To avoid clashes with other function it is
recommendable to add and remove the Toolbox with the following commands at
beginning and end of your scripts:

\todo[color=script!80,inline]{\ttfamily
addpath(genpath('C:/AMPER/VARToolbox'))

...

rmpath(genpath('C:/AMPER/VARToolbox'))}

To save Figures in high quality format, Ghostscript is needed\ (freely
available at \href{www.ghostscript.com}{www.ghostscript.com}). The VT 3.0
has been tested with Matlab R2016B on a Windows 10 machine.

\section{VAR Toolbox:\ High level description}

The VAR Toolbox is a collection of Matlab routines to perform VAR analysis.
Vector autoregressive models (VARs) are one of the most successful,
flexible, and easy to use models for the analysis of multivariate time
series. It is a natural extension of the univariate autoregressive model to
dynamic multivariate time series. In their well-known paper
\textquotedblleft Vector Autoregressions,\textquotedblright\ \cite%
{StockWatson2000} describe VAR models as especially useful (and successful)
tools for i) describing the dynamic behavior of economic and financial time
series and ii) for forecasting.

In addition to data description and forecasting, VAR models are also used
for iii) structural inference and iv) policy analysis. In structural
analysis, we generally need to impose certain assumptions about the causal
structure of the data under investigation. The resulting \textquotedblleft
structural\textquotedblright\ VAR\ model can then be used to analyze the
impact of unexpected shocks to specified variables on all the variables in
the model. This is normally done by means of {i{impulse responses}}, {%
forecast error variance decompositions}, and {historical decompositions}.

The VAR Toolbox allows for identification of structural shocks with zero
short-run restrictions; zero long-run restrictions; sign restrictions; and
with the external instrument approach (proxy SVAR). Impulse Response
Functions (IR), Forecast Error Variance Decomposition (VD), and Historical
Decompositions (HD) are computed according to the chosen identification.
Error bands are obtained with bootstrapping. The VAR Toolbox makes use of
few Matlab routines from the Econometrics Toolbox for Matlab by James P.
LeSage (freely available at \href{www.spatial-econometrics.com}{%
www.spatial-econometrics.com}).

It also includes a collection of Matlab routines that allows the user to
save and export high quality images from Matlab (using the Export\_fig
function by Oliver Woodford, freely available at \href{https://www.mathworks.com/matlabcentral/fileexchange/23629-export_fig}%
{https://www.mathworks.com/matlabcentral/fileexchange/23629-export\_fig}).
To enable this option, the Toolbox requires Ghostscript installed on your
computer (freely available at \href{www.ghostscript.com}{www.ghostscript.com}%
).

The VAR\ Toolbox is not meant to be efficient, but rather to be transparent
and allow the user to understand the econometrics of VARs step by step. The
codes are grouped in six categories (and respective folders):

\begin{itemize}
\item \textbf{Auxiliary}:\ codes that I borrowed from other public sources.
Each m-file has a reference to the original source.

\item \textbf{ExportFig}: this is a toolbox available at Oliver Woodford
website for exporting high quality figures. [add website]

\item \textbf{Figure}:\ codes for plotting high quality figures,
particularly thought for time series. For example, the functions in this
folder allow to efficiently add dates to the x-axis, to control the size and
font of Figures and the appearance of the legends, to plot charts with
shaded error bands, etc.

\item \textbf{Stats}: codes for the calculation of summary statistics,
moving correlations, pairwise correlations, etc.

\item \textbf{Utils}: codes that allow the smooth functioning if the Toolbox.

\item \textbf{VAR}:\ the codes for VAR estimation, identification,
computation of the impulse response functions, FEVD, HD.
\end{itemize}

The Matlab on examples page includes few examples on the functioning of the
VAR Toolbox.

\section{A Guided Example}

The idea of this manual is to explain the functioning of the VT\ by means of
a simple example -- the idea being that is much easier to learn by doing
rather than reading a technical manual and then go to the computer. As a
result, many of the functions included in the VT\ are not covered in this
manual, nor is a full description of the output of each function. Moreover,
I\ will need to stop every now and then to introduce some concepts and/or
notation. Sections that include technical details, derivations, etc will be
labelled with {\color{note} {\small {[Note]}}}, while sections that include
details on the practical example will be labelled with {%
\color{note} {\small 
{[Matlab]}}}.

Additional resources are available on my website:

\begin{itemize}
\item \href{https://sites.google.com/site/ambropo/replications}{%
https://sites.google.com/site/ambropo/replications} provide some lecture
notes on the basics of VARs that are a good complement to this manual.

\item \href{https://sites.google.com/site/ambropo/matlab-examples}{%
https://sites.google.com/site/ambropo/matlab-examples} provides a few
examples on how to estimate VARs with different identification schemes (in a
similar spirit to the example in this manual.

\item \href{https://sites.google.com/site/ambropo/replications}{%
https://sites.google.com/site/ambropo/replications}) provide the replication
codes for a few well-known VAR\ studies (e.g. \cite{StockWatson2000}, \cite%
{BlanchardQuah1989}, \cite{Uhlig2005}, and \cite{GertlerKaradi2015}).
\end{itemize}

I will start by introducing some (very light)\ notation.

\subsection{VARs: Basics {\color{note} {\protect\small {[Notes]}}\label%
{sec:basics}}}

Given a $k\times 1$ vector of time series ($x_{t}$) a Structural Vector
Autoregression (SVAR) of order $p$ is given by:%
\begin{equation}
x_{t}=\sum_{j=1}^{p}\Phi _{j}x_{t-j}+B\varepsilon _{t},
\label{eq:struct_var}
\end{equation}%
where $B$ is a $k\times k$ matrix and $\varepsilon _{t}$ is a $k\times 1$
vector of serially uncorrelated error terms, generally called
\textquotedblleft \emph{structural innovations}\textquotedblright\ or
\textquotedblleft \emph{structural shocks}\textquotedblright . All elements
of $\varepsilon _{t}$ are assumed to be mutually uncorrelated and $%
\varepsilon _{it}\sim i.i.d.(0,1)$. Note that the fact that the variance of
the structural shocks is equal to one is just a harmless normalization which
does not involve a loss of generality (as long as the diagonal elements of $A
$ remain unrestricted.\footnote{%
An alternative (and equivalently valid) normalization would be to leave
unrestricted the variance of the structural innovations, namely $\varepsilon
_{it}\sim i.i.d.(0,\sigma _{it})$ and assume tha the diagonal elements of $A$
to $1$.}

To keep the notation simple, consider a bivariate VAR(1), i.e. a VAR where
the number of variables is $k=2$ and the number of lags is $p=1$. This
simple bivariate VAR(1) can be written as a system of linear equations:%
\begin{equation}
\begin{bmatrix}
x_{1t} \\ 
x_{2t}%
\end{bmatrix}%
=%
\begin{bmatrix}
\phi _{11} & \phi _{12} \\ 
\phi _{21} & \phi _{22}%
\end{bmatrix}%
\begin{bmatrix}
x_{1t-1} \\ 
x_{2t-1}%
\end{bmatrix}%
+%
\begin{bmatrix}
b_{11} & b_{12} \\ 
b_{21} & b_{22}%
\end{bmatrix}%
\begin{bmatrix}
\varepsilon _{1t} \\ 
\varepsilon _{2t}%
\end{bmatrix}%
,  \label{eq:struct_var_1}
\end{equation}%
or:%
\begin{equation}
\begin{array}{c}
x_{1t}=\phi _{11}x_{1,t-1}+\phi _{12}x_{2,t-1}+b_{11}\varepsilon
_{1t}+b_{12}\varepsilon _{2t}, \\ 
x_{2t}=\phi _{21}x_{1,t-1}+\phi _{22}x_{2,t-1}+b_{21}\varepsilon
_{1t}+b_{22}\varepsilon _{2t},%
\end{array}
\label{eq:struct_var_2}
\end{equation}%
where $\varepsilon _{t}=(\varepsilon _{1t}^{\prime },\varepsilon
_{2t}^{\prime })^{\prime }$ is a $2\times 1$ vector of (unobservable)\
uncorrelated, zero mean, white noise processes. That is:%
\begin{equation}
\mathbb{V}(\varepsilon _{t})=\Sigma _{\varepsilon }=\left[ 
\begin{array}{cc}
1 & 0 \\ 
0 & 1%
\end{array}%
\right] =I.  \label{eq:struct_cov_1}
\end{equation}%
The assumption that the elements of $\varepsilon _{t}$ are mutually
uncorrelated is crucial. It implies that we can track the dynamic effects of
a shock to, say, $\varepsilon _{1t}$ to all variables in the VAR keeping the
other shock to zero (and vice versa). The $B$ matrix is also crucial.\ To
see that, consider a unit surprise in $\varepsilon _{1t}$. What are the
consequences for $x_{1t}$ and $x_{2t}$? The answer to this question is in
the first column of the $B$ matrix: $x_{1t}$ will increase by $b_{11}$and $%
x_{2t}$ will increase by $b_{21}$. This is why the $B$ matrix is also known
as the structural impact matrix. The $\Phi $ matrix can then be used to
track the dynamic effects in $t+1$, $t+2$, etc.

While all this sounds easy and great, there is a slight complication. The
structural innovations are unobservable. An econometrician can only estimate:%
\begin{equation}
\begin{array}{c}
x_{1t}=\phi _{11}x_{1,t-1}+\phi _{12}x_{2,t-1}+u_{1t}, \\ 
x_{2t}=\phi _{21}x_{1,t-1}+\phi _{22}x_{2,t-1}+u_{2t},%
\end{array}
\label{eq:red_var_1}
\end{equation}%
where the reduced-form innovations ($u_{t}$) are a linear combination of the
structural innovations:%
\begin{equation}
\begin{array}{c}
u_{1t}=b_{11}\varepsilon _{1t}+b_{12}\varepsilon _{2t}, \\ 
u_{2t}=b_{21}\varepsilon _{1t}+b_{22}\varepsilon _{2t}.%
\end{array}
\label{eq:red_resid_1}
\end{equation}%
The VAR in (\ref{eq:var_red_a}) is typically referred to as the \emph{%
reduced-form representation} of the structural VAR, which can be written
more compactly in matrix form as:%
\begin{equation}
\left[ 
\begin{array}{c}
x_{1t} \\ 
x_{2t}%
\end{array}%
\right] =%
\begin{bmatrix}
\phi _{11} & \phi _{12} \\ 
\phi _{21} & \phi _{22}%
\end{bmatrix}%
\left[ 
\begin{array}{c}
x_{1t-1} \\ 
x_{2t-1}%
\end{array}%
\right] +\left[ 
\begin{array}{c}
u_{1t} \\ 
u_{2t}%
\end{array}%
\right]  \label{eq:red_var_2}
\end{equation}%
where:%
\begin{equation}
\left[ 
\begin{array}{c}
u_{1t} \\ 
u_{2t}%
\end{array}%
\right] =%
\begin{bmatrix}
b_{11} & b_{12} \\ 
b_{21} & b_{22}%
\end{bmatrix}%
\begin{bmatrix}
\varepsilon _{1t} \\ 
\varepsilon _{2t}%
\end{bmatrix}%
,  \label{eq:red_resid_2}
\end{equation}%
or:%
\begin{equation}
x_{t}=\Phi x_{t-1}+u_{t}  \label{eq:red_var_3}
\end{equation}%
where $u_{t}=B\varepsilon _{t}$. Note that, in general, $\mathbf{\hat{\Sigma}%
}_{u}$, is a symmetric non-diagonal matrix:%
\begin{equation}
\mathbf{\hat{\Sigma}}_{u}=\left[ 
\begin{array}{cc}
\sigma _{1}^{2} & \sigma _{12}^{2} \\ 
- & \sigma _{2}^{2}%
\end{array}%
\right]  \label{eq:red_cov_1}
\end{equation}%
where its diagonal elements are the variances of the estimated reduced-form
error terms, $\sigma _{1}^{2}$ and $\sigma _{2}^{2}$; and the off-diagonal
elements are instead and the covariance between the estimated error terms ($%
\sigma _{12}=\sigma _{21}$).The covariance between the estimated reduced
form resiuduals plays an important role VARs because it collects the
information on the contemporaneous interaction of the variables in the
structural system, which (as we have just seen) is summarized by by the $B$
matrix. Indeed, using (\ref{eq_red_resid_a}) the covariance matrix of th
ereduced for residulas can be written as:%
\begin{equation}
\mathbf{\hat{\Sigma}}_{u}=\left[ 
\begin{array}{cc}
b_{11}^{2} & b_{11}b_{21}+b_{12}b_{22} \\ 
b_{11}b_{21}+b_{12}b_{22} & b_{22}^{2}%
\end{array}%
\right]  \label{eq:red_cov_2}
\end{equation}%
This shows that, differently from structural VARs, the reduced form
innovations are not informative about how shocks propagate throught the
system, as an innovation to $u_{1t}$ could be driven by either $\varepsilon
_{1t}$ or $\varepsilon _{2t}$ (and vice versa).

\subsection{Load \& Plot Data {\color{note} {\protect\small {[Matlab]}}}}

The example uses data on US industrial production ($IP_{t}$), consumer
prices ($CPI_{t}$), short-term interest rates\ ($R_{t}$), and the Excess
Bond Premium ($EBP_{t}$) from 1979:M7 to 2015:M3 (this is the data used by 
\cite{GertlerKaradi2015}). The code below shows a general way of loading the
data and managing it in a way that is consistent with the functioning of the
VT.

\todo[color=script!80,inline]{\ttfamily
\hspace{1mm}\textcolor{matlabgreen}{\%}\textcolor{matlabgreen}{\% 1. LOAD \& PLOT DATA }\\ 
\hspace{1mm}\textcolor{matlabgreen}{\%--------------------------------------------------------------------------  }\\ 
\hspace{1mm}\textcolor{matlabgreen}{\% Load  }\\ 
\hspace{1mm}[xlsdata, xlstext] = xlsread(\textcolor{matlabpurple}{'GK2015\_Data.xlsx'},\textcolor{matlabpurple}{'VAR\_data'}); \\ 
\hspace{1mm}data   = Num2NaN(xlsdata(:,3:end)); \\ 
\hspace{1mm}vnames = xlstext(1,3:end); \\ 
\hspace{1mm}\textcolor{matlabblue}{for} ii=1:length(vnames) \\ 
\hspace{1mm}\hspace{5mm} DATA.(vnames\{ii\}) = data(:,ii); \\ 
\hspace{1mm}\textcolor{matlabblue}{end} \\ 
\hspace{1mm}year = xlsdata(1,1); \\ 
\hspace{1mm}month = xlsdata(1,2); \\ 
\hspace{1mm}\textcolor{matlabgreen}{\% Observations }\\ 
\hspace{1mm}nobs = size(data,1); \\ 
\hspace{1mm}\textcolor{matlabgreen}{\% Set endogenous }\\ 
\hspace{1mm}VARvnames      = \{\textcolor{matlabpurple}{'gs1'},\textcolor{matlabpurple}{'logcpi'},\textcolor{matlabpurple}{'logip'},\textcolor{matlabpurple}{'ebp'}\}; \\ 
\hspace{1mm}VARvnames\_long = \{\textcolor{matlabpurple}{'Policy rate'},\textcolor{matlabpurple}{'CPI'},\textcolor{matlabpurple}{'Industrial Production'},\textcolor{matlabpurple}{'EBP'}\}; \\ 
\hspace{1mm}VARnvar        = length(VARvnames); \\ 
\hspace{1mm}\textcolor{matlabgreen}{\% Create matrices of variables to be used in the VAR }\\ 
\hspace{1mm}ENDO = nan(nobs,VARnvar); \\ 
\hspace{1mm}\textcolor{matlabblue}{for} ii=1:VARnvar \\ 
\hspace{1mm}\hspace{5mm} ENDO(:,ii) = DATA.(VARvnames\{ii\}); \\ 
\hspace{1mm}\textcolor{matlabblue}{end} \\ 
}

First the code reads from an Excel file and stores all data into the
structure \texttt{DATA}. The VAR Toolbox includes some functions that allow
to plot time series quickly and export them as high-quality PDFs, so that
they can be used directly in your papers. The code shows how to plot the
four time series in \texttt{ENDO}.

\todo[color=script!80,inline]{\ttfamily
\hspace{1mm}\textcolor{matlabgreen}{\% Open a figure of the desired size }\\ 
\hspace{1mm}FigSize(28,16) \\ 
\hspace{1mm}\textcolor{matlabblue}{for} ii=1:VARnvar \\ 
\hspace{1mm}\hspace{5mm} subplot(2,2,ii) \\ 
\hspace{1mm}\hspace{5mm} H(ii) = plot(ENDO(:,ii),\textcolor{matlabpurple}{'LineWidth'},2,\textcolor{matlabpurple}{'Color'},cmap(ii)); \\ 
\hspace{1mm}\hspace{5mm} title(VARvnames\_long(ii));  \\ 
\hspace{1mm}\hspace{5mm} DatesPlot(year+(month-1)/12,nobs,6,\textcolor{matlabpurple}{'m'}) \textcolor{matlabgreen}{\% Set the x-axis label  }\\ 
\hspace{1mm}\hspace{5mm} grid on;  \\ 
\hspace{1mm}\textcolor{matlabblue}{end} \\ 
\hspace{1mm}\textcolor{matlabgreen}{\% Legend }\\ 
\hspace{1mm}lopt = LegOption; lopt.handle = H; LegSubplot(VARvnames,lopt); FigFont(10); \\ 
\hspace{1mm}\textcolor{matlabgreen}{\% Save  }\\ 
\hspace{1mm}SaveFigure(\textcolor{matlabpurple}{'graphics/F1\_PLOT'},1) \\ 
}

Figure \ref{fig:F1_PLOT} reports the behavior of the interest rate on US
1-year Treasury bill, an index of industrial production, the CPI level and
the Excess Bond Premium (GZ) over the 1979:M7 to 2015:M3 sample period.

\begin{figure}[th]
\centering%
\begin{minipage}[b]{.9\textwidth}
\caption{\scshape{Raw Data}}\vspace{0.1cm}
\begin{center}
\includegraphics[width=\textwidth]{F1_PLOT}
\end{center}%
\footnotesize{{\scshape Note.} Raw data for interest rate on US 1-year Treasury bill, industrial production, CPI level and the Excess Bond Premium (GZ) from 1979;M7 to 2015:M3.}
\label{fig:F1_PLOT}
\end{minipage}
\end{figure}

Some useful functions are:

\begin{itemize}
\item \texttt{FigSize.m}: allows the user to choose the proportions of the
figure to plot. This is particularly useful when creating figures with many
panels.

\item \texttt{DatesPlot.m}: Adds dates to the horizontal axis of a chart (at
monthly, quarterly, and annual frequency) using a specified number of ticks.

\item \texttt{LegSubplot.m}: To be used in combination with the Matlab
built-in function subplot. \texttt{Legsubplot.m} creates a single legend for
the subplots, below the charts and centered.

\item \texttt{FigFont.m}: Sets the font of axes, title, legends, etc to a
specified font size.

\item \texttt{FigFont.m}: SaveFigure saves the chart in the selected format
(pdf, jpg, eps). The function allows the user to save the figure at high
quality standard using the \texttt{export\_fig.m} function created by Oliver
Woodford. Note that you need Ghostscript to be able to use this function.
\end{itemize}

\subsection{VAR\ estimation {\color{note} {\protect\small {[Matlab]}}}}

A\ VAR\ model can be estimated with a simple line of code, using the \texttt{%
VARmodel.m} function. To do that you need to specify a matrix including the
endogenous variables (\texttt{ENDO}), whether you want deterministic
variables, like a constant or a trend for example (\texttt{det}), and the
number of lags of the VAR (\texttt{nlags}). In the example, I\ specify a
simple bivariate VAR(12) in industrial production and interest rates, with a
constant.\footnote\textcolor{matlabgreen}{\% VAR ESTIMATION }\\ 
\hspace{1mm}\textcolor{matlabgreen}{\%--------------------------------------------------------------------------  }\\ 
\hspace{1mm}\textcolor{matlabgreen}{\% Set the deterministic variable in the VAR (1=constant, 2=trend) }\\ 
\hspace{1mm}det = 1; \\ 
\hspace{1mm}\textcolor{matlabgreen}{\% Set number of nlags }\\ 
\hspace{1mm}nlags = 12; \\ 
\hspace{1mm}\textcolor{matlabgreen}{\% Estimate VAR by OLS }\\ 
\hspace{1mm}[VAR, VARopt] = VARmodel(ENDO,nlags,det); \\ 
\hspace{1mm}disp(VAR) \\ 
\hspace{1mm}disp(VARopt) \\ 
\hspace{1mm}\textcolor{matlabgreen}{\% Add variable names to VARopt }\\ 
\hspace{1mm}VARopt.vnames = VARvnames; \\ 
\hspace{1mm}VARopt.figname= \textcolor{matlabpurple}{'graphics/'}; \\ 
\hspace{1mm}\textcolor{matlabgreen}{\% Print at screen and create table }\\ 
\hspace{1mm}[TABLE, beta] = VARprint(VAR,VARopt,2); \\ 
}

The cell array \texttt{VARvnames} defines the list of endogenous variables
that will be used to estimate the VAR model (in this case, industrial
production and interets rates, namely a subset of the data in \texttt{DATA}%
). The chosen data is then stored in the matrix \texttt{ENDO}. The
convention in the VT is that each column is a variable and each row is an
observation (with no missing observations allowed). That is:%
\begin{equation*}
\text{\texttt{ENDO}}=\left[ 
\begin{array}{cc}
IP_{1} & R_{1} \\ 
IP_{2} & R_{2} \\ 
... & ... \\ 
IP_{T} & R_{T}%
\end{array}%
\right] =\left( IP_{t}^{\prime },R_{t}^{\prime }\right) =x_{t}^{\prime }.
\end{equation*}%
This convention implies that, using the notation defined in the previous
section, \texttt{ENDO}$\ =x_{t}^{\prime }$. 

The VAR\ can then be estimated in a few lines of code.

\todo[color=script!80,inline]{\ttfamily
\hspace{1mm}\textcolor{matlabgreen}{\%}\textcolor{matlabgreen}{\% VAR ESTIMATION }\\ 
\hspace{1mm}\textcolor{matlabgreen}{\%--------------------------------------------------------------------------  }\\ 
\hspace{1mm}\textcolor{matlabgreen}{\% Set the deterministic variable in the VAR (1=constant, 2=trend) }\\ 
\hspace{1mm}det = 1; \\ 
\hspace{1mm}\textcolor{matlabgreen}{\% Set number of nlags }\\ 
\hspace{1mm}nlags = 12; \\ 
\hspace{1mm}\textcolor{matlabgreen}{\% Estimate VAR by OLS }\\ 
\hspace{1mm}[VAR, VARopt] = VARmodel(ENDO,nlags,det); \\ 
\hspace{1mm}disp(VAR) \\ 
\hspace{1mm}disp(VARopt) \\ 
\hspace{1mm}\textcolor{matlabgreen}{\% Add variable names to VARopt }\\ 
\hspace{1mm}VARopt.vnames = VARvnames; \\ 
\hspace{1mm}VARopt.figname= \textcolor{matlabpurple}{'graphics/'}; \\ 
\hspace{1mm}\textcolor{matlabgreen}{\% Print at screen and create table }\\ 
\hspace{1mm}[TABLE, beta] = VARprint(VAR,VARopt,2); \\ 
}

The results of the VAR estimation are stored in the structures \texttt{VAR}
and \texttt{VARopt}. The structure \texttt{VAR}\ includes all the estimation
results. These can be seen by executing the command \colorbox{script!80}{%
\texttt{disp(VARprint(VAR)}} which prints the following output in the
command window:

\setlength{\parindent}{.2cm}
\begin{verbatim}
>> disp(VAR)
         ENDO: [396x2 double]
         nlag: 12
        const: 1
         EXOG: []
         nobs: 384
         nvar: 4
      nvar_ex: 0
      nlag_ex: 0
       ncoeff: 24
    ntotcoeff: 25
          eq1: [1x1 struct]
          eq2: [1x1 struct]
          eq3: [1x1 struct]
          eq4: [1x1 struct]
           Ft: [49x4 double]
            F: [4x49 double]
        sigma: [4x4 double]
        resid: [384x4 double]
            X: [384x49 double]
            Y: [384x4 double]
        Fcomp: [48x48 double]
       maxEig: 0.9974
           Fp: [4x4x12 double]
            B: []
      BfromSR: []
          PSI: []
\end{verbatim}

\setlength{\parindent}{.0cm}

The structure \texttt{VAR} includes all the inputs to the \texttt{VARmodel.m}
function, such as the matrix of endogenous variables (\texttt{VAR.ENDO}),
the number of lags (\texttt{VAR.nlags}), and the number of endogenous
variables\ (\texttt{VAR.nvar}). But also includes the estimation output. For
example:

\begin{itemize}
\item The matrix \texttt{VAR.F} collects the estimated coefficients
following the notation in (\ref{eq:var_red_b}), namely we have that \texttt{%
VAR.F}$=F$. For a VAR with $12$ lags and $2$ variables plus a constant, this
means that \texttt{VAR.F} is a $2\times (12\times 2+1)$ matrix.

\item The covariance matrix of the VAR residuals defined by (\ref%
{eq:var_red_cov}) is instead stored in \texttt{VAR.sigma}$=\Sigma _{u}$, of
size $2\times 2$.

\item Note that the structural impact matrix \texttt{VAR.B}$\ =B$, whihc we
defined in equation (), is empty.\ This is because, for the moment we
estimated only the reduced form VAR (1). The next sections will show how,
with additional assumptions, the also the strutcural form of the VAR can be
recovered.
\end{itemize}

Other outputs are the OLS\ equation-by-equation estimation results
(structures \texttt{VAR.eq}), the VAR\ companion matrix (\texttt{VAR.Fcomp}%
), the maximum eigenvalue of the VAR\ (\texttt{VAR.maxEig}), etc.\ 

The structure \texttt{VARopt} includes a few auxiliary variables that are
created automatically by the \texttt{VARmodel.m} function and will be needed
below for the calculation of impulse responses, variance decompositions,
etc. The variables stored in \texttt{VARopt} can be seen by executing the
command \colorbox{script!80}{\texttt{disp(VARopt)}}, which prints the
following output in the Matlab command window:
\begin{verbatim}
>> disp(VARopt)
       vnames: []
    vnames_ex: []
       snames: []
       nsteps: 40
       impact: 0
         shut: 0
        ident: 'oir'
       recurs: 'wold'
       ndraws: 100
         pctg: 95
       method: 'bs'
         pick: 0
      quality: 0
     suptitle: 0
    firstdate: []
    frequency: 'q'
      figname: []
\end{verbatim}

These variables include the number of steps for impulse response functions
and variance decompositions (\texttt{nsteps}), the labels of the endogenous
or exogenous variables for plots (\texttt{vnames} and \texttt{vnames\_ex}),
the confidence levels for the computation of error bands (\texttt{pctg}),
etc. While some variables are automatically created by the VARmodel
function, some other variables need to be inputted by the user.\ For example:

\begin{itemize}
\item \colorbox{script!80}{\texttt{VARopt.vnames = VARvnames}} stores in 
\texttt{VARopt} the endogenous variables' names.

\item \colorbox{script!80}{\texttt{VARopt.figname =
'graphics/'}} stores in \texttt{VARopt} the name of the folder where all
figures will be saved.
\end{itemize}

So that when executing \colorbox{script!80}{\texttt{disp(VARopt)}} we now
get:
\begin{verbatim}
>> disp(VARopt)
       vnames: {'gs1'  'logcpi'  'logip'  'ebp'}
    vnames_ex: []
       snames: []
       nsteps: 40
       impact: 0
         shut: 0
        ident: 'oir'
       recurs: 'wold'
       ndraws: 100
         pctg: 95
       method: 'bs'
         pick: 0
      quality: 0
     suptitle: 0
    firstdate: []
    frequency: 'q'
      figname: 'graphics/'
\end{verbatim}

\subsection{The Identification\ Problem {\color{note} {\protect\small
{[Notes]}}}}

In the previous section we have seen how to estimate a reduced form VAR.
Ignoring lagged variables beyond order 1 for ease of notation, we estimated
the following:%
\begin{equation}
\begin{bmatrix}
IP_{t} \\ 
R_{t}%
\end{bmatrix}%
=\left[ 
\begin{array}{cc}
\phi _{11} & \phi _{12} \\ 
\phi _{21} & \phi _{22}%
\end{array}%
\right] 
\begin{bmatrix}
IP_{t-1} \\ 
R_{t-1}%
\end{bmatrix}%
+\text{other lags}+%
\begin{bmatrix}
u_{t}^{IP} \\ 
u_{t}^{R}%
\end{bmatrix}%
,  \label{eq:red_2var}
\end{equation}%
Now, imagine that you are asked to estimate the effect of a moneatry policy
shock to industrial production and consumer prices. Unfortunately, the
reduced form innovation to the interets rate ($u_{t}^{R}$) is not going tpo
help us. The reason is that, as we discussed in Section \ref{sec:basics}, $%
u_{t}^{R}$ is a linear combination of the true structural shocks in the
economy. So, it does not tell us anything about how monetary opoicy affects
output and prices. 

To see that more clearly, assume that the `true' model of the economy is
given by the following structural VAR:%
\begin{equation}
\begin{bmatrix}
IP_{t} \\ 
R_{t}%
\end{bmatrix}%
=\text{all lags}+\left[ 
\begin{array}{cc}
b_{11} & b_{12} \\ 
b_{21} & b_{22}%
\end{array}%
\right] 
\begin{bmatrix}
\varepsilon _{t}^{Demand} \\ 
\varepsilon _{t}^{Mon.\ Pol}%
\end{bmatrix}%
,  \label{eq:struct_2var}
\end{equation}%
where the matrix $B$ and the structural shocks $\varepsilon $ are
unobserved. The SVAR\ in (XX) assumes that time series of industrial
production and interest rates are driven by a combination of demand and
moneatry policy shocks.\footnote{%
Again this is not a very realistic assumption, but it simplifies the math
that follows. a more realistic VAR\ woudl ave included more varibles and
more shocks.} It is obvious that the reduced form innovation to the interets
rate, $u_{t}^{R}$, is a linear combination of all shocks, and not just the
monetary policy shock.

To answer the question of what are the effects of monetary policy on the
economy, we need to find the values of the $B$ matrix. This is known as the
identification problem. For example, the coefficients $b_{11}$ and $b_{21}$
give us the impact effect of moneatry policy on all variables. The matrix of
coefficient $\Phi $, whihc we estimated in the reduced form VAR, can then be
used to trace out the dynamic effects of moneatry policy on the economy
beyond the impact effect.

So, how can we go from the reduced form representation to the structural
representation of the VAR? We have seen abpove that we know that $%
u=B\varepsilon $, so that we can write:%
\begin{equation}
\hat{\Sigma}_{u}=E\left[ \hat{u}_{t}\hat{u}_{t}^{\prime }\right] =E\left[
B\varepsilon _{t}\left( B\varepsilon _{t}\right) ^{\prime }\right] =B\Sigma
_{\varepsilon }B^{\prime }=BB^{\prime }.  \label{eq:red2struct_1}
\end{equation}%
where remember that $\Sigma _{\varepsilon }=I$. This means that there is a
mapping between the estimated covariance matrix of the reduced form
residuals ($\hat{\Sigma}_{u}$)\ and the unobserved matrix of structural
impact coefficients. We can think of (\ref{eq:2}) as a system of nonlinear
equations in the $4$ unknown coefficients of the $B$ matrix. The problem the 
$\Sigma _{u}$ matrix, given its symmetric nature, leads to only $3$
independent restrictions. In other words, we have 
\begin{equation}
\left[ 
\begin{array}{cc}
\sigma _{1}^{2} & \sigma _{12}^{2} \\ 
- & \sigma _{2}^{2}%
\end{array}%
\right] =\left[ 
\begin{array}{cc}
b_{11} & b_{12} \\ 
b_{21} & b_{22}%
\end{array}%
\right] \left[ 
\begin{array}{cc}
b_{11} & b_{21} \\ 
b_{12} & b_{22}%
\end{array}%
\right] ,  \label{eq:red2struct_2}
\end{equation}%
which can be rewritten as the following system of equations:%
\begin{equation}
\begin{array}{l}
\sigma _{1}^{2}=b_{11}^{2}+b_{12}^{2} \\ 
\sigma _{12}^{2}=b_{11}b_{21}+b_{12}b_{22} \\ 
\sigma _{12}^{2}=b_{11}b_{21}+b_{12}b_{22} \\ 
\sigma _{2}^{2}=b_{21}^{2}+b_{22}^{2}%
\end{array}
\label{eq:red2struct_3}
\end{equation}%
Note that, becasue of the symmetry of the $\Sigma _{u}$ matrix, the second
and the third equation are identical. This means that we are left with $4$
unknowns (the $b$'s) but only $3$ equations. The system is clearly
under-identified, meaning that we need additional conditions if we want to
recover the structural parameters. 

There are many ways of identifying solving the identificartion problem
described above. In the following section, I will describe a few of the most
popular ones, and how thy can be implemented in the VAR\ Tolbox.

\subsection{Identification by zero contemporaneous resrictions}

Identification using zero contemporaneous resrictions (also known as
recursive identification, as it will be clear in a second) were developed by
Sims1980, and are by far the most commonly used identification scheme used
in the literature. In a recursive SVAR, identification is achieved by
assuming that some shocks have zero contemporaneous effect on some endoenous
variables. This amounts to setting some of the non-diagonal elements of the $%
B$ matrix to zero -- therefore reducing the number of unknown coefficients.

Typically, it is assumed that the first variable in the system is only
affected by the first strcutral shock, the second is contemporaneously
affected by the first and second structural shock, and so on. In our
example, that means to assume that the structural VAR\ is 
\begin{equation}
\begin{bmatrix}
IP_{t} \\ 
R_{t}%
\end{bmatrix}%
=\text{all lags}+\left[ 
\begin{array}{cc}
b_{11} & 0 \\ 
b_{21} & b_{22}%
\end{array}%
\right] 
\begin{bmatrix}
\varepsilon _{t}^{Demand} \\ 
\varepsilon _{t}^{Mon.\ Pol}%
\end{bmatrix}%
,
\end{equation}%
where note that the industrial production is not contemporaneously affected
by the monetary policy shock (while interest rates are contemporaneously
affected by both the demand and the moneatry policuy shock). This assumption
could be justified by the fact that moneatry policy takes time to affect
real variables like industrial production.

What are the implications for the identificariton problem descrobed above?
The simple answer is that we now have 3 instead of 4 parameters to estimate,
and 3 restrictions implied by the reduced form covariance matrix. That is,
the system of equations (\ref{eq:red2struct_3}) now becomes:%
\begin{equation}
\begin{array}{l}
\sigma _{1}^{2}=b_{11}^{2} \\ 
\sigma _{12}^{2}=b_{11}b_{21} \\ 
\sigma _{2}^{2}=b_{21}^{2}+b_{22}^{2}%
\end{array}
\label{eq:red2struct_4}
\end{equation}%
which can be easily solved to get:%
\begin{equation*}
\begin{array}{c}
b_{11}=\sigma _{1}, \\ 
b_{21}=\sigma _{12}^{2}/\sigma _{1}, \\ 
b_{22}=\sqrt{\sigma _{2}^{2}-\left( \sigma _{12}^{2}/\sigma _{1}^{2}\right)
^{2}}%
\end{array}%
\end{equation*}%
The VAR\ is identified! This means that it is possible to compute the \emph{%
impact} impulse response of all endogenous variables by simply looking at
the estimated B\ matrix. For example, consider a one standard deviation
shock to moneatry policy, i.e. $\varepsilon _{t}^{Mon.\ Pol}=1$. Using the
structural VAR\ representation we get:%
\begin{equation}
\begin{array}{l}
\mathcal{IR}_{IP,0}^{Mon.\ Pol}=0 \\ 
\mathcal{IR}_{R,0}^{Mon.\ Pol}=\sigma _{12}^{2}/\sigma _{1}%
\end{array}
\label{eq:IR_example}
\end{equation}%
where $\mathcal{IR}_{i,0}^{j}$ denotes the impulse response at horizon $0$
(i.e. on impact), of variable $i$ to the structural shock $j$. Knowing the
impact response it is then easy to compute the dynamic response of all
endogenous variables (i.e. at horoizons $>0$) with the reduced form $\Phi $
matrix. 

In summary, we have:%
\begin{equation}
\begin{array}{l}
\mathcal{IR}_{0}=B\varepsilon _{t} \\ 
\mathcal{IR}_{h}=\Phi ^{h}\mathcal{IR}_{0}^{j}\ \ \ \ \ \ h>0%
\end{array}
\label{eq:IR_recusive}
\end{equation}

\subsection{Variance decompositions}

\subsection{Historical decompositions}

\bibliographystyle{chicago}
\bibliography{BIBLIO}

\appendix

\section{Appendix}

\subsection{The Identification\ Problem {\color{note} {\protect\small
{[Notes]}}}}

In the previous section we have seen how to estimate a reduced form VAR.
Ignoring lagged variables beyond order 1 for ease of notation, we estimated
the following:%
\begin{equation}
\begin{bmatrix}
R_{t} \\ 
IP_{t} \\ 
CPI_{t} \\ 
EBP_{t}%
\end{bmatrix}%
=\left[ 
\begin{array}{cccc}
\phi _{11} & \phi _{12} & \phi _{13} & \phi _{14} \\ 
\phi _{21} & \phi _{22} & \phi _{23} & \phi _{24} \\ 
\phi _{31} & \phi _{12} & \phi _{33} & \phi _{34} \\ 
\phi _{41} & \phi _{22} & \phi _{43} & \phi _{44}%
\end{array}%
\right] 
\begin{bmatrix}
R_{t-1} \\ 
IP_{t-1} \\ 
CPI_{t-1} \\ 
EBP_{t-1}%
\end{bmatrix}%
+\text{other lags}+%
\begin{bmatrix}
u_{t}^{R} \\ 
u_{t}^{IP} \\ 
u_{t}^{CPI} \\ 
u_{t}^{EBP}%
\end{bmatrix}%
,
\end{equation}%
Now, imagine that you are asked to estimate the effect of a moneatry policy
shock to industrial production and consumer prices. Unfortunately, the
reduced form innovation to the interets rate ($u_{t}^{R}$) is not going tpo
help us. The reason is that, as we discussed in Section \ref{sec:basics}, $%
u_{t}^{R}$ is a linear combination of the true structural shocks in the
economy. So, it does not tell us anything about how monetary opoicy affects
output and prices. 

To see that more clearly, assume that the `true' model of the economy is
given by the following structural VAR:%
\begin{equation}
\begin{bmatrix}
R_{t} \\ 
IP_{t} \\ 
CPI_{t} \\ 
EBP_{t}%
\end{bmatrix}%
=\text{all lags}+\left[ 
\begin{array}{cccc}
b_{11} & b_{12} & b_{13} & b_{14} \\ 
b_{21} & b_{22} & b_{23} & b_{24} \\ 
b_{31} & b_{12} & b_{33} & b_{34} \\ 
b_{41} & b_{22} & b_{43} & b_{44}%
\end{array}%
\right] 
\begin{bmatrix}
\varepsilon _{t}^{Mon.\ Pol} \\ 
\varepsilon _{t}^{Demand} \\ 
\varepsilon _{t}^{Supply} \\ 
\varepsilon _{t}^{Financial}%
\end{bmatrix}%
,  \label{eq:GK_VAR}
\end{equation}%
where the matrix $B$ and the structural shocks $\varepsilon $ are
unobserved. The SVAR\ in (XX) assumes that time series of interest rates,
industrial production, consumer prices and the excess bond premium are
driven by a combination of moneatry, demand, supply and financial shocks. It
is obvious that the reduced form innovation to the interets rate, $u_{t}^{R}$%
, is a linear combination of all shocks, and not just the monetary policy
shock.

To answer the question of what are the effects of monetary policy on the
economy, we need to find the values of the $B$ matrix. This is known as the
identification problem. For example, the coefficients $b_{11}$, $b_{21}$, $%
b_{31}$, and $b_{41}$ give us the impact effect of moneatry policy on all
variables. The matrix of coefficient $\Phi $, whihc we estimated in the
reduced form VAR, can then be used to trace out the dynamic effects of
moneatry policy on the economy beyond the impact effect.

So, how can we go from the reduced form representation to the structural
representation of the VAR? From equation (\ref{eq:var_reduced_c}) we know
that:%
\begin{equation}
\hat{\Sigma}_{u}=E\left[ \hat{u}_{t}\hat{u}_{t}^{\prime }\right] =E\left[
B\varepsilon \left( B\varepsilon \right) ^{\prime }\right] =B\Sigma
_{\varepsilon }B^{\prime }=BB^{\prime }.  \label{eq:2}
\end{equation}%
where remember that $\Sigma _{\varepsilon }=I$. This means that there is a
mapping between the estimated covariance matrix of the reduced form
residuals ($\hat{\Sigma}_{u}$)\ and the unobserved matrix of structural
impact coefficients. We can think of (\ref{eq:2}) as a system of nonlinear
equations in the $4\times 4$ unknown coefficients of the $B$ matrix. The
problem the $\Sigma _{u}$ matrix ---given its symmetric nature--- would
contain only $4+(4\times 3)/2$ parameters. In other words, we have 
\begin{equation*}
\mathbf{\Sigma }_{u}=\left[ 
\begin{array}{cccc}
\sigma _{1}^{2} & \sigma _{12}^{2} & \sigma _{13}^{2} & \sigma _{14}^{2} \\ 
- & \sigma _{2}^{2} & \sigma _{23}^{2} & \sigma _{24}^{2} \\ 
- & - & \sigma _{3}^{2} & \sigma _{34}^{2} \\ 
- & - & - & \sigma _{4}^{2}%
\end{array}%
\right] =\left[ 
\begin{array}{cccc}
b_{11} & b_{12} & b_{13} & b_{14} \\ 
b_{21} & b_{22} & b_{23} & b_{24} \\ 
b_{31} & b_{12} & b_{33} & b_{34} \\ 
b_{41} & b_{22} & b_{43} & b_{44}%
\end{array}%
\right] \left[ 
\begin{array}{cccc}
b_{11} & b_{12} & b_{13} & b_{14} \\ 
b_{21} & b_{22} & b_{23} & b_{24} \\ 
b_{31} & b_{12} & b_{33} & b_{34} \\ 
b_{41} & b_{22} & b_{43} & b_{44}%
\end{array}%
\right] ^{\prime }
\end{equation*}%
which shows that there are 16 unknowns but only 10 independent equations.
The system is clearly under-identified, meaning that we need additional
conditions if we want to recover the structural parameters. 

There are many ways of identifying solving the identificartion problem
described above. In the following section, I will describe a few of the most
popular ones, and how thy can be implemented in the VAR\ Tolbox.

\subsection{Impulse responses}

ZERO LONG-RUN RESTRICTIONS. Similarly to the short-run restrictions,
identification is achieved by making the assumption that some variables of
the VAR\ cannot affect some other variables in the long-run. Specifically we
will assume that the first variable is not affected in the long run by the
others; the second is affected in the long run by the first variable but not
by the others, and so on and so forth.

SIGN RESTRICTIONS. Identification is achieved by restricting the sign of the
responses of selected model variables to structural shocks, using economic
theory as a guidance

\end{document}
